\begin{table}[H]
\begin{tabular}{cl}
\textbf{5.0} & \natline{ओं श्री परमात्मने नमः} \\
 & \natline{अथ पञ्चमोऽध्यायः} \\
 & \natline{कर्मसन्न्यासयोगः}
\end{tabular}
\end{table}

\begin{table}[H]
\begin{tabular}{cl}
\textbf{5.1} & \natline{अर्जुन उवाच} \\
 & \natline{सन्न्यासं कर्मणां कृष्ण} \\
 & \natline{पुनर्योगं च शंससि |} \\
 & \natline{यच्छ्रेय एतयोरेकं} \\
 & \natline{तन्मे ब्रूहि सुनिश्चितम् ||}
\end{tabular}
\end{table}

\begin{table}[H]
\begin{tabular}{cl}
\textbf{5.2} & \natline{श्री भगवानुवाच} \\
 & \natline{सन्न्यासः कर्मयोगश्च} \\
 & \natline{निश्श्रेयसकरावुभौ |} \\
 & \natline{तयोस्तु कर्मसन्न्यासात्} \\
 & \natline{कर्मयोगो विशिष्यते ||}
\end{tabular}
\end{table}

\begin{table}[H]
\begin{tabular}{cl}
\textbf{5.3} & \natline{ज्ञेयः स नित्यसन्न्यासी} \\
 & \natline{यो न द्वेष्टि न काङ्क्षति |} \\
 & \natline{निर्द्वन्द्वो हि महाबाहो} \\
 & \natline{सुखं बन्धात्प्रमुच्यते ||}
\end{tabular}
\end{table}

\begin{table}[H]
\begin{tabular}{cl}
\textbf{5.4} & \natline{साङ्ख्ययोगौ पृथग्बालाः} \\
 & \natline{प्रवदन्ति न पण्डिताः |} \\
 & \natline{एकमप्यास्थितः सम्यक्} \\
 & \natline{उभयोर्विन्दते फलम् ||}
\end{tabular}
\end{table}

\begin{table}[H]
\begin{tabular}{cl}
\textbf{5.5} & \natline{यत्साङ्ख्यैः प्राप्यते स्थानं} \\
 & \natline{तद्योगैरपि गम्यते |} \\
 & \natline{एकं साङ्ख्यं च योगं च} \\
 & \natline{यः पश्यति स पश्यति ||}
\end{tabular}
\end{table}

\begin{table}[H]
\begin{tabular}{cl}
\textbf{5.6} & \natline{सन्न्यासस्तु महाबाहो} \\
 & \natline{दुःखमाप्तुमयोगतः |} \\
 & \natline{योगयुक्तो मुनिर्ब्रह्म} \\
 & \natline{नचिरेणाधिगच्छति ||}
\end{tabular}
\end{table}

\begin{table}[H]
\begin{tabular}{cl}
\textbf{5.7} & \natline{योगयुक्तो विशुद्धात्मा} \\
 & \natline{विजितात्मा जितेन्द्रियः |} \\
 & \natline{सर्वभूतात्मभूतात्मा} \\
 & \natline{कुर्वन्नपि न लिप्यते ||}
\end{tabular}
\end{table}

\begin{table}[H]
\begin{tabular}{cl}
\textbf{5.8} & \natline{नैव किञ्चित्करोमीति} \\
 & \natline{युक्तो मन्येत तत्त्ववित् |} \\
 & \natline{पश्यन्शृण्वन् स्पृशञ्जिघ्रन्} \\
 & \natline{अश्नन्गच्छन्स्वपन्श्वसन् ||}
\end{tabular}
\end{table}

\begin{table}[H]
\begin{tabular}{cl}
\textbf{5.9} & \natline{प्रलपन् विसृजन् गृह्णन्} \\
 & \natline{उन्मिषन्निमिषन्नपि |} \\
 & \natline{इन्द्रियाणीन्द्रियार्थेषु} \\
 & \natline{वर्तन्त इति धारयन् ||}
\end{tabular}
\end{table}

\begin{table}[H]
\begin{tabular}{cl}
\textbf{5.10} & \natline{ब्रह्मण्याधाय कर्माणि} \\
 & \natline{सङ्गं त्यक्त्वा करोति यः |} \\
 & \natline{लिप्यते न स पापेन} \\
 & \natline{पद्मपत्रमिवाम्भसा ||}
\end{tabular}
\end{table}

\begin{table}[H]
\begin{tabular}{cl}
\textbf{5.11} & \natline{कायेन मनसा बुद्ध्या} \\
 & \natline{केवलैरिन्द्रियैरपि |} \\
 & \natline{योगिनः कर्म कुर्वन्ति} \\
 & \natline{सङ्गं त्यक्त्वात्मशुद्धये ||}
\end{tabular}
\end{table}

\begin{table}[H]
\begin{tabular}{cl}
\textbf{5.12} & \natline{युक्तः कर्मफलं त्यक्त्वा} \\
 & \natline{शान्तिमाप्नोति नैष्ठिकीम् |} \\
 & \natline{अयुक्तः कामकारेण} \\
 & \natline{फले सक्तो निबध्यते ||}
\end{tabular}
\end{table}

\begin{table}[H]
\begin{tabular}{cl}
\textbf{5.13} & \natline{सर्वकर्माणि मनसा} \\
 & \natline{सन्न्यस्यास्ते सुखं वशी |} \\
 & \natline{नवद्वारे पुरे देही} \\
 & \natline{नैव कुर्वन्न कारयन् ||}
\end{tabular}
\end{table}

\begin{table}[H]
\begin{tabular}{cl}
\textbf{5.14} & \natline{न कर्तृत्वं न कर्माणि} \\
 & \natline{लोकस्य सृजति प्रभुः |} \\
 & \natline{न कर्मफलसंयोगं} \\
 & \natline{स्वभावस्तु प्रवर्तते ||}
\end{tabular}
\end{table}

\begin{table}[H]
\begin{tabular}{cl}
\textbf{5.15} & \natline{नादत्ते कस्यचित्पापं} \\
 & \natline{न चैव सुकृतं विभुः |} \\
 & \natline{अज्ञानेनावृतं ज्ञानं} \\
 & \natline{तेन मुह्यन्ति जन्तवः ||}
\end{tabular}
\end{table}

\begin{table}[H]
\begin{tabular}{cl}
\textbf{5.16} & \natline{ज्ञानेन तु तदज्ञानं} \\
 & \natline{येषां नाशितमात्मनः |} \\
 & \natline{तेषामादित्यवज्ज्ञानं} \\
 & \natline{प्रकाशयति तत्परम् ||}
\end{tabular}
\end{table}

\begin{table}[H]
\begin{tabular}{cl}
\textbf{5.17} & \natline{तद्बुद्धयस्तदात्मानः} \\
 & \natline{तन्निष्ठास्तत्परायणाः |} \\
 & \natline{गच्छन्त्यपुनरावृत्तिं} \\
 & \natline{ज्ञाननिर्धूतकल्मषाः ||}
\end{tabular}
\end{table}

\begin{table}[H]
\begin{tabular}{cl}
\textbf{5.18} & \natline{विद्याविनयसम्पन्ने} \\
 & \natline{ब्राह्मणे गवि हस्तिनि |} \\
 & \natline{शुनि चैव श्वपाके च} \\
 & \natline{पण्डिताः समदर्शिनः ||}
\end{tabular}
\end{table}

\begin{table}[H]
\begin{tabular}{cl}
\textbf{5.19} & \natline{इहैव तैर्जितः सर्गः} \\
 & \natline{येषां साम्ये स्थितं मनः |} \\
 & \natline{निर्दोषं हि समं ब्रह्म} \\
 & \natline{तस्मात् ब्रह्मणि ते स्थिताः ||}
\end{tabular}
\end{table}

\begin{table}[H]
\begin{tabular}{cl}
\textbf{5.20} & \natline{न प्रहृष्येत्प्रियं प्राप्य} \\
 & \natline{नोद्विजेत्प्राप्य चाप्रियम् |} \\
 & \natline{स्थिरबुद्धिरसम्मूढः} \\
 & \natline{ब्रह्मवित् ब्रह्मणि स्थितः ||}
\end{tabular}
\end{table}

\begin{table}[H]
\begin{tabular}{cl}
\textbf{5.21} & \natline{बाह्यस्पर्शेष्वसक्तात्मा} \\
 & \natline{विन्दत्यात्मनि यत् सुखम् |} \\
 & \natline{स ब्रह्मयोगयुक्तात्मा} \\
 & \natline{सुखमक्षयमश्नुते ||}
\end{tabular}
\end{table}

\begin{table}[H]
\begin{tabular}{cl}
\textbf{5.22} & \natline{ये हि संस्पर्शजा भोगाः} \\
 & \natline{दुःखयोनय एव ते |} \\
 & \natline{आद्यन्तवन्तः कौन्तेय} \\
 & \natline{न तेषु रमते बुधः ||}
\end{tabular}
\end{table}

\begin{table}[H]
\begin{tabular}{cl}
\textbf{5.23} & \natline{शक्नोतीहैव यः सोढुं} \\
 & \natline{प्राक् शरीरविमोक्षणात् |} \\
 & \natline{कामक्रोधोद्भवं वेगं} \\
 & \natline{स युक्तः स सुखी नरः ||}
\end{tabular}
\end{table}

\begin{table}[H]
\begin{tabular}{cl}
\textbf{5.24} & \natline{योऽन्तःसुखोऽन्तरारामः} \\
 & \natline{तथान्तर्ज्योतिरेव यः |} \\
 & \natline{स योगी ब्रह्मनिर्वाणं} \\
 & \natline{ब्रह्मभूतोऽधिगच्छति ||}
\end{tabular}
\end{table}

\begin{table}[H]
\begin{tabular}{cl}
\textbf{5.25} & \natline{लभन्ते ब्रह्मनिर्वाणम्} \\
 & \natline{ऋषयः क्षीणकल्मषाः |} \\
 & \natline{छिन्नद्वैधा यतात्मानः} \\
 & \natline{सर्वभूतहिते रताः ||}
\end{tabular}
\end{table}

\begin{table}[H]
\begin{tabular}{cl}
\textbf{5.26} & \natline{कामक्रोधवियुक्तानां} \\
 & \natline{यतीनां यतचेतसाम् |} \\
 & \natline{अभितो ब्रह्मनिर्वाणं} \\
 & \natline{वर्तते विदितात्मनाम् ||}
\end{tabular}
\end{table}

\begin{table}[H]
\begin{tabular}{cl}
\textbf{5.27} & \natline{स्पर्शान् कृत्वा बहिर्बाह्यान्} \\
 & \natline{चक्षुश्चैवान्तरे भ्रुवोः |} \\
 & \natline{प्राणापानौ समौ कृत्वा} \\
 & \natline{नासाभ्यन्तरचारिणौ ||}
\end{tabular}
\end{table}

\begin{table}[H]
\begin{tabular}{cl}
\textbf{5.28} & \natline{यतेन्द्रियमनोबुद्धिः} \\
 & \natline{मुनिर्मोक्षपरायणः |} \\
 & \natline{विगतेच्छाभयक्रोधः} \\
 & \natline{यः सदा मुक्त एव सः ||}
\end{tabular}
\end{table}

\begin{table}[H]
\begin{tabular}{cl}
\textbf{5.29} & \natline{भोक्तारं यज्ञतपसां} \\
 & \natline{सर्वलोकमहेश्वरम् |} \\
 & \natline{सुहृदं सर्वभूतानां} \\
 & \natline{ज्ञात्वा मां शान्तिमृच्छति। ||}
\end{tabular}
\end{table}

