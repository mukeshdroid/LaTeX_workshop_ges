\begin{table}[H]
\begin{tabular}{cl}
\textbf{4.0} & \natline{ओं श्री परमात्मने नमः} \\
 & \natline{अथ चतुर्थोऽध्यायः} \\
 & \natline{ज्ञानयोगः}
\end{tabular}
\end{table}

\begin{table}[H]
\begin{tabular}{cl}
\textbf{4.1} & \natline{श्री भगवानुवाच} \\
 & \natline{इमं विवस्वते योगं} \\
 & \natline{प्रोक्तवानहमव्ययम् |} \\
 & \natline{विवस्वान्मनवे प्राह} \\
 & \natline{मनुरिक्ष्वाकवेऽब्रवीत् ||}
\end{tabular}
\end{table}

\begin{table}[H]
\begin{tabular}{cl}
\textbf{4.2} & \natline{एवं परम्पराप्राप्तम्} \\
 & \natline{इमं राजर्षयो विदुः |} \\
 & \natline{स कालेनेह महता} \\
 & \natline{योगो नष्टः परन्तप ||}
\end{tabular}
\end{table}

\begin{table}[H]
\begin{tabular}{cl}
\textbf{4.3} & \natline{स एवायं मया तेऽद्य} \\
 & \natline{योगः प्रोक्तः पुरातनः |} \\
 & \natline{भक्तोऽसि मे सखा चेति} \\
 & \natline{रहस्यं ह्येतदुत्तमम् ||}
\end{tabular}
\end{table}

\begin{table}[H]
\begin{tabular}{cl}
\textbf{4.4} & \natline{अर्जुन उवाच} \\
 & \natline{अपरं भवतो जन्म} \\
 & \natline{परं जन्म विवस्वतः |} \\
 & \natline{कथमेतद्विजानीयां} \\
 & \natline{त्वमादौ प्रोक्तवानिति ||}
\end{tabular}
\end{table}

\begin{table}[H]
\begin{tabular}{cl}
\textbf{4.5} & \natline{श्री भगवानुवाच} \\
 & \natline{बहूनि मे व्यतीतानि} \\
 & \natline{जन्मानि तव चार्जुन |} \\
 & \natline{तान्यहं वेद सर्वाणि} \\
 & \natline{न त्वं वेत्थ परन्तप ||}
\end{tabular}
\end{table}

\begin{table}[H]
\begin{tabular}{cl}
\textbf{4.6} & \natline{अजोऽपि सन्नव्ययात्मा} \\
 & \natline{भूतानामीश्वरोऽपि सन् |} \\
 & \natline{प्रकृतिं स्वामधिष्ठाय} \\
 & \natline{सम्भवाम्यात्ममायया ||}
\end{tabular}
\end{table}

\begin{table}[H]
\begin{tabular}{cl}
\textbf{4.7} & \natline{यदा यदा हि धर्मस्य} \\
 & \natline{ग्लानिर्भवति भारत |} \\
 & \natline{अभ्युत्थानमधर्मस्य} \\
 & \natline{तदाऽऽत्मानं सृजाम्यहम् ||}
\end{tabular}
\end{table}

\begin{table}[H]
\begin{tabular}{cl}
\textbf{4.8} & \natline{परित्राणाय साधूनां} \\
 & \natline{विनाशाय च दुष्कृताम् |} \\
 & \natline{धर्मसंस्थापनार्थाय} \\
 & \natline{सम्भवामि युगे युगे ||}
\end{tabular}
\end{table}

\begin{table}[H]
\begin{tabular}{cl}
\textbf{4.9} & \natline{जन्म कर्म च मे दिव्यम्} \\
 & \natline{एवं यो वेत्ति तत्त्वतः |} \\
 & \natline{त्यक्त्वा देहं पुनर्जन्म} \\
 & \natline{नैति मामेति सोऽर्जुन ||}
\end{tabular}
\end{table}

\begin{table}[H]
\begin{tabular}{cl}
\textbf{4.10} & \natline{वीतरागभयक्रोधाः} \\
 & \natline{मन्मया मामुपाश्रिताः |} \\
 & \natline{बहवो ज्ञानतपसा} \\
 & \natline{पूता मद्भावमागताः ||}
\end{tabular}
\end{table}

\begin{table}[H]
\begin{tabular}{cl}
\textbf{4.11} & \natline{ये यथा मां प्रपद्यन्ते} \\
 & \natline{तांस्तथैव भजाम्यहम् |} \\
 & \natline{मम वर्त्मानुवर्तन्ते} \\
 & \natline{मनुष्याः पार्थ सर्वशः ||}
\end{tabular}
\end{table}

\begin{table}[H]
\begin{tabular}{cl}
\textbf{4.12} & \natline{काङ्क्षन्तः कर्मणां सिद्धिं} \\
 & \natline{यजन्त इह देवताः |} \\
 & \natline{क्षिप्रं हि मानुषे लोके} \\
 & \natline{सिद्धिर्भवति कर्मजा ||}
\end{tabular}
\end{table}

\begin{table}[H]
\begin{tabular}{cl}
\textbf{4.13} & \natline{चातुर्वर्ण्यं मया सृष्टं} \\
 & \natline{गुणकर्मविभागशः |} \\
 & \natline{तस्य कर्तारमपि मां} \\
 & \natline{विद्ध्यकर्तारमव्ययम् ||}
\end{tabular}
\end{table}

\begin{table}[H]
\begin{tabular}{cl}
\textbf{4.14} & \natline{न मां कर्माणि लिम्पन्ति} \\
 & \natline{न मे कर्मफले स्पृहा |} \\
 & \natline{इति मां योऽभिजानाति} \\
 & \natline{कर्मभिर्न स बध्यते ||}
\end{tabular}
\end{table}

\begin{table}[H]
\begin{tabular}{cl}
\textbf{4.15} & \natline{एवं ज्ञात्वा कृतं कर्म} \\
 & \natline{पूर्वैरपि मुमुक्षुभिः |} \\
 & \natline{कुरु कर्मैव तस्मात्त्वं} \\
 & \natline{पूर्वैः पूर्वतरं कृतम् ||}
\end{tabular}
\end{table}

\begin{table}[H]
\begin{tabular}{cl}
\textbf{4.16} & \natline{किं कर्म किमकर्मेति} \\
 & \natline{कवयोऽप्यत्र मोहिताः |} \\
 & \natline{तत्ते कर्म प्रवक्ष्यामि} \\
 & \natline{यज्ज्ञात्वा मोक्ष्यसेऽशुभात् ||}
\end{tabular}
\end{table}

\begin{table}[H]
\begin{tabular}{cl}
\textbf{4.17} & \natline{कर्मणो ह्यपि बोद्धव्यं} \\
 & \natline{बोद्धव्यं च विकर्मणः |} \\
 & \natline{अकर्मणश्च बोद्धव्यं} \\
 & \natline{गहना कर्मणो गतिः ||}
\end{tabular}
\end{table}

\begin{table}[H]
\begin{tabular}{cl}
\textbf{4.18} & \natline{कर्मण्यकर्म यः पश्येत्} \\
 & \natline{अकर्मणिच कर्म यः |} \\
 & \natline{स बुद्धिमान्मनुष्येषु} \\
 & \natline{स युक्तः कृत्स्नकर्मकृत् ||}
\end{tabular}
\end{table}

\begin{table}[H]
\begin{tabular}{cl}
\textbf{4.19} & \natline{यस्य सर्वे समारम्भाः} \\
 & \natline{कामसङ्कल्पवर्जिताः |} \\
 & \natline{ज्ञानाग्निदग्धकर्माणं} \\
 & \natline{तमाहुः पण्डितं बुधाः ||}
\end{tabular}
\end{table}

\begin{table}[H]
\begin{tabular}{cl}
\textbf{4.20} & \natline{त्यक्त्वा कर्मफलासङ्गं} \\
 & \natline{नित्यतृप्तो निराश्रयः |} \\
 & \natline{कर्मण्यभिप्रवृत्तोऽपि} \\
 & \natline{नैव किञ्चित्करोति सः ||}
\end{tabular}
\end{table}

\begin{table}[H]
\begin{tabular}{cl}
\textbf{4.21} & \natline{निराशीर्यतचित्तात्मा} \\
 & \natline{त्यक्तसर्व परिग्रहः |} \\
 & \natline{शारीरं केवलं कर्म} \\
 & \natline{कुर्वन्नाप्नोति किल्बिषम् ||}
\end{tabular}
\end{table}

\begin{table}[H]
\begin{tabular}{cl}
\textbf{4.22} & \natline{यदृच्छालाभसन्तुष्टः} \\
 & \natline{द्वन्द्वातीतो विमत्सरः |} \\
 & \natline{समः सिद्धावसिद्धौ च} \\
 & \natline{कृत्वापि न निबध्यते ||}
\end{tabular}
\end{table}

\begin{table}[H]
\begin{tabular}{cl}
\textbf{4.23} & \natline{गतसङ्गस्य मुक्तस्य} \\
 & \natline{ज्ञानावस्थितचेतसः |} \\
 & \natline{यज्ञायाचरतः कर्म} \\
 & \natline{समग्रं प्रविलीयते ||}
\end{tabular}
\end{table}

\begin{table}[H]
\begin{tabular}{cl}
\textbf{4.24} & \natline{ब्रह्मार्पणं ब्रह्म हविः} \\
 & \natline{ब्रह्माग्नौ ब्रह्मणा हुतम् |} \\
 & \natline{ब्रह्मैव तेन गन्तव्यं} \\
 & \natline{ब्रह्मकर्मसमाधिना ||}
\end{tabular}
\end{table}

\begin{table}[H]
\begin{tabular}{cl}
\textbf{4.25} & \natline{दैवमेवापरे यज्ञं} \\
 & \natline{योगिनः पर्युपासते |} \\
 & \natline{ब्रह्माग्नावपरे यज्ञं} \\
 & \natline{यज्ञेनैवोपजुह्वति ||}
\end{tabular}
\end{table}

\begin{table}[H]
\begin{tabular}{cl}
\textbf{4.26} & \natline{श्रोत्रादीनीन्द्रियाण्यन्ये} \\
 & \natline{संयमाग्निषु जुह्वति |} \\
 & \natline{शब्दादीन्विषयानन्ये} \\
 & \natline{इन्द्रियाग्निषु जुह्वति ||}
\end{tabular}
\end{table}

\begin{table}[H]
\begin{tabular}{cl}
\textbf{4.27} & \natline{सर्वाणीन्द्रियकर्माणि} \\
 & \natline{प्राणकर्माणि चापरे |} \\
 & \natline{आत्मसंयमयोगाग्नौ} \\
 & \natline{जुह्वति ज्ञानदीपिते ||}
\end{tabular}
\end{table}

\begin{table}[H]
\begin{tabular}{cl}
\textbf{4.28} & \natline{द्रव्ययज्ञास्तपोयज्ञाः} \\
 & \natline{योगयज्ञास्तथाऽपरे |} \\
 & \natline{स्वाध्यायज्ञानयज्ञाश्च} \\
 & \natline{यतयः संशितव्रताः ||}
\end{tabular}
\end{table}

\begin{table}[H]
\begin{tabular}{cl}
\textbf{4.29} & \natline{अपाने जुह्वति प्राणं} \\
 & \natline{प्राणेऽपानं तथापरे |} \\
 & \natline{प्राणापानगती रुद्ध्वा} \\
 & \natline{प्राणायामपरायणाः ||}
\end{tabular}
\end{table}

\begin{table}[H]
\begin{tabular}{cl}
\textbf{4.30} & \natline{अपरे नियताहाराः} \\
 & \natline{प्राणान्प्राणेषु जुह्वति |} \\
 & \natline{सर्वेऽप्येते यज्ञविदः} \\
 & \natline{यज्ञक्षपितकल्मषाः ||}
\end{tabular}
\end{table}

\begin{table}[H]
\begin{tabular}{cl}
\textbf{4.31} & \natline{यज्ञशिष्टामृतभुजः} \\
 & \natline{यान्ति ब्रह्म सनातनम् |} \\
 & \natline{नायं लोकोऽस्त्ययज्ञस्य} \\
 & \natline{कुतोऽन्यः कुरुसत्तम ||}
\end{tabular}
\end{table}

\begin{table}[H]
\begin{tabular}{cl}
\textbf{4.32} & \natline{एवं बहुविधा यज्ञाः} \\
 & \natline{वितता ब्रह्मणो मुखे |} \\
 & \natline{कर्मजान्विद्धि तान्सर्वान्} \\
 & \natline{एवं ज्ञात्वा विमोक्ष्यसे ||}
\end{tabular}
\end{table}

\begin{table}[H]
\begin{tabular}{cl}
\textbf{4.33} & \natline{श्रेयान्द्रव्यमयाद्यज्ञात्} \\
 & \natline{ज्ञानयज्ञः परन्तप |} \\
 & \natline{सर्वं कर्माखिलं पार्थ} \\
 & \natline{ज्ञाने परिसमाप्यते ||}
\end{tabular}
\end{table}

\begin{table}[H]
\begin{tabular}{cl}
\textbf{4.34} & \natline{तद्विद्धि प्रणिपातेन} \\
 & \natline{परिप्रश्नेन सेवया |} \\
 & \natline{उपदेक्ष्यन्ति ते ज्ञानं} \\
 & \natline{ज्ञानिनस्तत्त्वदर्शिनः ||}
\end{tabular}
\end{table}

\begin{table}[H]
\begin{tabular}{cl}
\textbf{4.35} & \natline{यज्ज्ञात्वा न पुनर्मोहम्} \\
 & \natline{एवं यास्यसि पाण्डव |} \\
 & \natline{येन भूतान्यशेषेण} \\
 & \natline{द्रक्ष्यस्यात्मन्यथो मयि ||}
\end{tabular}
\end{table}

\begin{table}[H]
\begin{tabular}{cl}
\textbf{4.36} & \natline{अपि चेदसि पापेभ्यः} \\
 & \natline{सर्वेभ्यः पापकृत्तमः |} \\
 & \natline{सर्वं ज्ञानप्लवेनैव} \\
 & \natline{वृजिनं सन्तरिष्यसि ||}
\end{tabular}
\end{table}

\begin{table}[H]
\begin{tabular}{cl}
\textbf{4.37} & \natline{यथैधांसि समिद्धोऽग्निः} \\
 & \natline{भस्मसात्कुरुतेऽर्जुन |} \\
 & \natline{ज्ञानाग्निः सर्वकर्माणि} \\
 & \natline{भस्मसात्कुरुते तथा ||}
\end{tabular}
\end{table}

\begin{table}[H]
\begin{tabular}{cl}
\textbf{4.38} & \natline{न हि ज्ञानेन सदृशं} \\
 & \natline{पवित्रमिह विद्यते |} \\
 & \natline{तत्स्वयं योगसंसिद्धः} \\
 & \natline{कालेनात्मनि विन्दति ||}
\end{tabular}
\end{table}

\begin{table}[H]
\begin{tabular}{cl}
\textbf{4.39} & \natline{श्रद्धावान् लभते ज्ञानं} \\
 & \natline{तत्परः संयतेन्द्रियः |} \\
 & \natline{ज्ञानं लब्ध्वा परां शान्तिम्} \\
 & \natline{अचिरेणाधिगच्छति ||}
\end{tabular}
\end{table}

\begin{table}[H]
\begin{tabular}{cl}
\textbf{4.40} & \natline{अज्ञश्चाश्रद्दधानश्च} \\
 & \natline{संशयात्मा विनश्यति |} \\
 & \natline{नायं लोकोऽस्ति न परः} \\
 & \natline{न सुखं संशयात्मनः ||}
\end{tabular}
\end{table}

\begin{table}[H]
\begin{tabular}{cl}
\textbf{4.41} & \natline{योगसन्न्यस्तकर्माणं} \\
 & \natline{ज्ञानसञ्छिन्नसंशयम् |} \\
 & \natline{आत्मवन्तं न कर्माणि} \\
 & \natline{निबध्नन्ति धनञ्जय ||}
\end{tabular}
\end{table}

\begin{table}[H]
\begin{tabular}{cl}
\textbf{4.42} & \natline{तस्मादज्ञानसम्भूतं} \\
 & \natline{हृत्स्थं ज्ञानासिनात्मनः |} \\
 & \natline{छित्त्वैनं संशयं योगम्} \\
 & \natline{आतिष्ठोत्तिष्ठ भारत ||}
\end{tabular}
\end{table}

