\begin{table}[H]
\begin{tabular}{cl}
\textbf{3.0} & \natline{ओं श्री परमात्मने नमः} \\
 & \natline{अथ तृतीयोऽध्यायः} \\
 & \natline{कर्मयोगः}
\end{tabular}
\end{table}

\begin{table}[H]
\begin{tabular}{cl}
\textbf{3.1} & \natline{अर्जुन उवाच} \\
 & \natline{ज्यायसी चेत्कर्मणस्ते} \\
 & \natline{मता बुद्धिर्जनार्दन |} \\
 & \natline{तत्किं कर्मणि घोरे मां} \\
 & \natline{नियोजयसि केशव ||}
\end{tabular}
\end{table}

\begin{table}[H]
\begin{tabular}{cl}
\textbf{3.2} & \natline{व्यामिश्रेणेव वाक्येन} \\
 & \natline{बुद्धिं मोहयसीव मे |} \\
 & \natline{तदेकं वद निश्चित्य} \\
 & \natline{येन श्रेयोऽहमाप्नुयाम् ||}
\end{tabular}
\end{table}

\begin{table}[H]
\begin{tabular}{cl}
\textbf{3.3} & \natline{स्री भगवानुवाच} \\
 & \natline{लोकेऽस्मिन्द्विविधा निष्ठा} \\
 & \natline{पुरा प्रोक्ता मयाऽनघ |} \\
 & \natline{ज्ञानयोगेन साङ्ख्यानां} \\
 & \natline{कर्मयोगेन योगिनाम् ||}
\end{tabular}
\end{table}

\begin{table}[H]
\begin{tabular}{cl}
\textbf{3.4} & \natline{न कर्मणामनारम्भात्} \\
 & \natline{नैष्कर्म्यं पुरुषोऽश्नुते |} \\
 & \natline{न च सन्न्यसनादेव} \\
 & \natline{सिद्धिं समधिगच्छति ||}
\end{tabular}
\end{table}

\begin{table}[H]
\begin{tabular}{cl}
\textbf{3.5} & \natline{न हि कश्चित्क्षणमपि} \\
 & \natline{जातु तिष्ठत्यकर्मकृत् |} \\
 & \natline{कार्यते ह्यवशः कर्म} \\
 & \natline{सर्वः प्रकृतिजैर्गुणैः ||}
\end{tabular}
\end{table}

\begin{table}[H]
\begin{tabular}{cl}
\textbf{3.6} & \natline{कर्मेन्द्रियाणि संयम्य} \\
 & \natline{य आस्ते मनसा स्मरन् |} \\
 & \natline{इन्द्रियार्थान्विमूढात्मा} \\
 & \natline{मिथ्याचारः स उच्यते ||}
\end{tabular}
\end{table}

\begin{table}[H]
\begin{tabular}{cl}
\textbf{3.7} & \natline{यस्त्विन्द्रियाणि मनसा} \\
 & \natline{नियम्यारभतेऽर्जुन |} \\
 & \natline{कर्मेन्द्रियैः कर्मयोगम्} \\
 & \natline{असक्तः स विशिष्यते ||}
\end{tabular}
\end{table}

\begin{table}[H]
\begin{tabular}{cl}
\textbf{3.8} & \natline{नियतं कुरु कर्म त्वं} \\
 & \natline{कर्म ज्यायो ह्यकर्मणः |} \\
 & \natline{शरीरयात्राऽपि च ते} \\
 & \natline{न प्रसिद्ध्येदकर्मणः ||}
\end{tabular}
\end{table}

\begin{table}[H]
\begin{tabular}{cl}
\textbf{3.9} & \natline{यज्ञार्थात्कर्मणोऽन्यत्र} \\
 & \natline{लोकोऽयं कर्मबन्धनः |} \\
 & \natline{तदर्थं कर्म कौन्तेय} \\
 & \natline{मुक्तसङ्गः समाचर ||}
\end{tabular}
\end{table}

\begin{table}[H]
\begin{tabular}{cl}
\textbf{3.10} & \natline{सहयज्ञाः प्रजाः सृष्ट्वा} \\
 & \natline{पुरोवाच प्रजापतिः |} \\
 & \natline{अनेन प्रसविष्यध्वं} \\
 & \natline{एष वोऽस्त्विष्टकामधुक् ||}
\end{tabular}
\end{table}

\begin{table}[H]
\begin{tabular}{cl}
\textbf{3.11} & \natline{देवान्भावयताऽनेन} \\
 & \natline{ते देवा भावयन्तु वः |} \\
 & \natline{परस्परं भावयन्तः} \\
 & \natline{श्रेयः परमवाप्स्यथ ||}
\end{tabular}
\end{table}

\begin{table}[H]
\begin{tabular}{cl}
\textbf{3.12} & \natline{इष्टान्भोगान्हि वो देवाः} \\
 & \natline{दास्यन्ते यज्ञभाविताः |} \\
 & \natline{तैर्दत्तानप्रदायैभ्यः} \\
 & \natline{यो भुङ्क्ते स्तेन एव सः ||}
\end{tabular}
\end{table}

\begin{table}[H]
\begin{tabular}{cl}
\textbf{3.13} & \natline{यज्ञशिष्टाशिनः सन्तः} \\
 & \natline{मुच्यन्ते सर्वकिल्बिषैः |} \\
 & \natline{भुञ्जते ते त्वघं पापाः} \\
 & \natline{ये पचन्त्यात्मकारणात् ||}
\end{tabular}
\end{table}

\begin{table}[H]
\begin{tabular}{cl}
\textbf{3.14} & \natline{अन्नाद्भवन्ति भूतानि} \\
 & \natline{पर्जन्यादन्नसम्भवः |} \\
 & \natline{यज्ञाद्भवति पर्जन्यः} \\
 & \natline{यज्ञः कर्मसमुद्भवः ||}
\end{tabular}
\end{table}

\begin{table}[H]
\begin{tabular}{cl}
\textbf{3.15} & \natline{कर्म ब्रह्मोद्भवं विद्धि} \\
 & \natline{ब्रह्माक्षरसमुद्भवम् |} \\
 & \natline{तस्मात्सर्वगतं ब्रह्म} \\
 & \natline{नित्यं यज्ञे प्रतिष्ठितम् ||}
\end{tabular}
\end{table}

\begin{table}[H]
\begin{tabular}{cl}
\textbf{3.16} & \natline{एवं प्रवर्तितं चक्रं} \\
 & \natline{नानुवर्तयतीह यः |} \\
 & \natline{अघायुरिन्द्रियारामः} \\
 & \natline{मोघं पार्थ स जीवति ||}
\end{tabular}
\end{table}

\begin{table}[H]
\begin{tabular}{cl}
\textbf{3.17} & \natline{यस्त्वात्मरतिरेव स्यात्} \\
 & \natline{आत्मतृप्तश्च मानवः |} \\
 & \natline{आत्मन्येव च सन्तुष्टः} \\
 & \natline{तस्य कार्यं न विद्यते ||}
\end{tabular}
\end{table}

\begin{table}[H]
\begin{tabular}{cl}
\textbf{3.18} & \natline{नैव तस्य कृतेनार्थः} \\
 & \natline{नाकृतेनेह कश्चन |} \\
 & \natline{न चास्य सर्वभूतेषु} \\
 & \natline{कश्चिदर्थव्यपाश्रयः ||}
\end{tabular}
\end{table}

\begin{table}[H]
\begin{tabular}{cl}
\textbf{3.19} & \natline{तस्मादसक्तः सततं} \\
 & \natline{कार्यं कर्म समाचर |} \\
 & \natline{असक्तो ह्याचरन्कर्म} \\
 & \natline{परमाप्नोति पूरुषः ||}
\end{tabular}
\end{table}

\begin{table}[H]
\begin{tabular}{cl}
\textbf{3.20} & \natline{कर्मणैव हि संसिद्धिं} \\
 & \natline{आस्थिता जनकादयः |} \\
 & \natline{लोकसङ्ग्रहमेवापि} \\
 & \natline{सम्पश्यन्कर्तुमर्हसि ||}
\end{tabular}
\end{table}

\begin{table}[H]
\begin{tabular}{cl}
\textbf{3.21} & \natline{यद्यदाचरति श्रेष्ठः} \\
 & \natline{तत्तदेवेतरो जनः |} \\
 & \natline{स यत्प्रमाणं कुरुते} \\
 & \natline{लोकस्तदनुवर्तते ||}
\end{tabular}
\end{table}

\begin{table}[H]
\begin{tabular}{cl}
\textbf{3.22} & \natline{न मे पार्थास्ति कर्तव्यं} \\
 & \natline{त्रिषु लोकेषु किञ्चन |} \\
 & \natline{नानवाप्तमवाप्तव्यं} \\
 & \natline{वर्त एव च कर्मणि ||}
\end{tabular}
\end{table}

\begin{table}[H]
\begin{tabular}{cl}
\textbf{3.23} & \natline{यदि ह्यहं न वर्तेय} \\
 & \natline{जातु कर्मण्यतन्द्रितः |} \\
 & \natline{मम वर्त्मानुवर्तन्ते} \\
 & \natline{मनुष्याः पार्थ सर्वशः ||}
\end{tabular}
\end{table}

\begin{table}[H]
\begin{tabular}{cl}
\textbf{3.24} & \natline{उत्सीदेयुरिमे लोकाः} \\
 & \natline{न कुर्यां कर्म चेदहम् |} \\
 & \natline{सङ्करस्य च कर्ता स्याम्} \\
 & \natline{उपहन्यामिमाः प्रजाः ||}
\end{tabular}
\end{table}

\begin{table}[H]
\begin{tabular}{cl}
\textbf{3.25} & \natline{सक्ताः कर्मण्यविद्वांसः} \\
 & \natline{यथा कुर्वन्ति भारत |} \\
 & \natline{कुर्याद्विद्वांस्तथाऽसक्तः} \\
 & \natline{चिकीर्षुर्लोकसङ्ग्रहम् ||}
\end{tabular}
\end{table}

\begin{table}[H]
\begin{tabular}{cl}
\textbf{3.26} & \natline{न बुद्धिभेदं जनयेत्} \\
 & \natline{अज्ञानां कर्मसङ्गिनाम् |} \\
 & \natline{जोषयेत्सर्वकर्माणि} \\
 & \natline{विद्वान्युक्तः समाचरन् ||}
\end{tabular}
\end{table}

\begin{table}[H]
\begin{tabular}{cl}
\textbf{3.27} & \natline{प्रकृतेः क्रियमाणानि} \\
 & \natline{गुणैः कर्माणि सर्वशः |} \\
 & \natline{अहङ्कारविमूढात्मा} \\
 & \natline{कर्ताऽहमिति मन्यते ||}
\end{tabular}
\end{table}

\begin{table}[H]
\begin{tabular}{cl}
\textbf{3.28} & \natline{तत्त्ववित्तु महाबाहो} \\
 & \natline{गुणकर्मविभागयोः |} \\
 & \natline{गुणा गुणेषु वर्तन्ते} \\
 & \natline{इति मत्वा न सज्जते ||}
\end{tabular}
\end{table}

\begin{table}[H]
\begin{tabular}{cl}
\textbf{3.29} & \natline{प्रकृतेर्गुणसम्मूढाः} \\
 & \natline{सज्जन्ते गुणकर्मसु |} \\
 & \natline{तानकृत्स्नविदो मन्दान्} \\
 & \natline{कृत्स्नविन्न विचालयेत् ||}
\end{tabular}
\end{table}

\begin{table}[H]
\begin{tabular}{cl}
\textbf{3.30} & \natline{मयि सर्वाणि कर्माणि} \\
 & \natline{सन्न्यस्याध्यात्मचेतसा |} \\
 & \natline{निराशीर्निर्ममो भूत्वा} \\
 & \natline{युध्यस्व विगतज्वरः ||}
\end{tabular}
\end{table}

\begin{table}[H]
\begin{tabular}{cl}
\textbf{3.31} & \natline{ये मे मतमिदं नित्यम्} \\
 & \natline{अनुतिष्ठन्ति मानवाः |} \\
 & \natline{श्रद्धावन्तोऽनसूयन्तः} \\
 & \natline{मुच्यन्ते तेऽपि कर्मभिः ||}
\end{tabular}
\end{table}

\begin{table}[H]
\begin{tabular}{cl}
\textbf{3.32} & \natline{ये त्वेतदभ्यसूयन्तः} \\
 & \natline{नानुतिष्ठन्ति मे मतम् |} \\
 & \natline{सर्वज्ञानविमूढांस्तान्} \\
 & \natline{विद्धि नष्टानचेतसः ||}
\end{tabular}
\end{table}

\begin{table}[H]
\begin{tabular}{cl}
\textbf{3.33} & \natline{सदृशं चेष्टते स्वस्याः} \\
 & \natline{प्रकृतेर्ज्ञानवानपि |} \\
 & \natline{प्रकृतिं यान्ति भूतानि} \\
 & \natline{निग्रहः किं करिष्यति ||}
\end{tabular}
\end{table}

\begin{table}[H]
\begin{tabular}{cl}
\textbf{3.34} & \natline{इन्द्रियस्येन्द्रियस्यार्थे} \\
 & \natline{रागद्वेषौ व्यवस्थितौ |} \\
 & \natline{तयोर्न वशमागच्छेत्} \\
 & \natline{तौ ह्यस्य परिपन्थिनौ ||}
\end{tabular}
\end{table}

\begin{table}[H]
\begin{tabular}{cl}
\textbf{3.35} & \natline{श्रेयान्स्वधर्मो विगुणः} \\
 & \natline{परधर्मात्स्वनुष्ठितात् |} \\
 & \natline{स्वधर्मे निधनं श्रेयः} \\
 & \natline{परधर्मो भयावहः ||}
\end{tabular}
\end{table}

\begin{table}[H]
\begin{tabular}{cl}
\textbf{3.36} & \natline{अर्जुन उवाच} \\
 & \natline{अथ केन प्रयुक्तोऽयं} \\
 & \natline{पापं चरति पूरुषः |} \\
 & \natline{अनिच्छन्नपि वार्ष्णेय} \\
 & \natline{बलादिव नियोजितः ||}
\end{tabular}
\end{table}

\begin{table}[H]
\begin{tabular}{cl}
\textbf{3.37} & \natline{श्री भगवानुवाच} \\
 & \natline{काम एष क्रोध एषः} \\
 & \natline{रजोगुणसमुद्भवः |} \\
 & \natline{महाशनो महापाप्मा} \\
 & \natline{विद्ध्येनमिह वैरिणम् ||}
\end{tabular}
\end{table}

\begin{table}[H]
\begin{tabular}{cl}
\textbf{3.38} & \natline{धूमेनाव्रियते वह्निः} \\
 & \natline{यथाऽदर्शो मलेन च |} \\
 & \natline{यथोल्बेनावृतो गर्भः} \\
 & \natline{तथा तेनेदमावृतम् ||}
\end{tabular}
\end{table}

\begin{table}[H]
\begin{tabular}{cl}
\textbf{3.39} & \natline{आवृतं ज्ञानमेतेन} \\
 & \natline{ज्ञानिनो नित्यवैरिणा |} \\
 & \natline{कामरूपेण कौन्तेय} \\
 & \natline{दुष्पूरेणानलेन च ||}
\end{tabular}
\end{table}

\begin{table}[H]
\begin{tabular}{cl}
\textbf{3.40} & \natline{इन्द्रियाणि मनो बुद्धिः} \\
 & \natline{अस्याधिष्ठानमुच्यते |} \\
 & \natline{एतैर्विमोहयत्येषः} \\
 & \natline{ज्ञानमावृत्य देहिनम् ||}
\end{tabular}
\end{table}

\begin{table}[H]
\begin{tabular}{cl}
\textbf{3.41} & \natline{तस्मात्त्वमिन्द्रियाण्यादौ} \\
 & \natline{नियम्य भरतर्षभ |} \\
 & \natline{पाप्मानं प्रजहि ह्येनं} \\
 & \natline{ज्ञानविज्ञाननाशनम् ||}
\end{tabular}
\end{table}

\begin{table}[H]
\begin{tabular}{cl}
\textbf{3.42} & \natline{इन्द्रियाणि पराण्याहुः} \\
 & \natline{इन्द्रियेभ्यः परं मनः |} \\
 & \natline{मनसस्तु पराबुद्धिः} \\
 & \natline{यो बुद्धेः परतस्तु सः ||}
\end{tabular}
\end{table}

\begin{table}[H]
\begin{tabular}{cl}
\textbf{3.43} & \natline{एवं बुद्धेः परं बुद्ध्वा} \\
 & \natline{संस्तभ्यात्मानमात्मना |} \\
 & \natline{जहि शत्रुं महाबाहो} \\
 & \natline{कामरूपं दुरासदम् ||}
\end{tabular}
\end{table}

