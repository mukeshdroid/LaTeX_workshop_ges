\begin{table}[H]
\begin{tabular}{cl}
\textbf{5.0} & \natline{ఓం శ్రీ పరమాత్మనే నమః} \\
 & \natline{అథ పఞ్చమోఽధ్యాయః} \\
 & \natline{కర్మసన్న్యాసయోగః}
\end{tabular}
\end{table}

\begin{table}[H]
\begin{tabular}{cl}
\textbf{5.1} & \natline{అర్జున ఉవాచ} \\
 & \natline{సన్న్యాసం కర్మణాం కృష్ణ} \\
 & \natline{పునర్యోగం చ శంససి |} \\
 & \natline{యచ్ఛ్రేయ ఏతయోరేకం} \\
 & \natline{తన్మే బ్రూహి సునిశ్చితమ్ ||}
\end{tabular}
\end{table}

\begin{table}[H]
\begin{tabular}{cl}
\textbf{5.2} & \natline{శ్రీ భగవానువాచ} \\
 & \natline{సన్న్యాసః కర్మయోగశ్చ} \\
 & \natline{నిశ్శ్రేయసకరావుభౌ |} \\
 & \natline{తయోస్తు కర్మసన్న్యాసాత్} \\
 & \natline{కర్మయోగో విశిష్యతే ||}
\end{tabular}
\end{table}

\begin{table}[H]
\begin{tabular}{cl}
\textbf{5.3} & \natline{జ్ఞేయః స నిత్యసన్న్యాసీ} \\
 & \natline{యో న ద్వేష్టి న కాఙ్క్షతి |} \\
 & \natline{నిర్ద్వన్ద్వో హి మహాబాహో} \\
 & \natline{సుఖం బన్ధాత్ప్రముచ్యతే ||}
\end{tabular}
\end{table}

\begin{table}[H]
\begin{tabular}{cl}
\textbf{5.4} & \natline{సాఙ్ఖ్యయోగౌ పృథగ్బాలాః} \\
 & \natline{ప్రవదన్తి న పణ్డితాః |} \\
 & \natline{ఏకమప్యాస్థితః సమ్యక్} \\
 & \natline{ఉభయోర్విన్దతే ఫలమ్ ||}
\end{tabular}
\end{table}

\begin{table}[H]
\begin{tabular}{cl}
\textbf{5.5} & \natline{యత్సాఙ్ఖ్యైః ప్రాప్యతే స్థానం} \\
 & \natline{తద్యోగైరపి గమ్యతే |} \\
 & \natline{ఏకం సాఙ్ఖ్యం చ యోగం చ} \\
 & \natline{యః పశ్యతి స పశ్యతి ||}
\end{tabular}
\end{table}

\begin{table}[H]
\begin{tabular}{cl}
\textbf{5.6} & \natline{సన్న్యాసస్తు మహాబాహో} \\
 & \natline{దుఃఖమాప్తుమయోగతః |} \\
 & \natline{యోగయుక్తో మునిర్బ్రహ్మ} \\
 & \natline{నచిరేణాధిగచ్ఛతి ||}
\end{tabular}
\end{table}

\begin{table}[H]
\begin{tabular}{cl}
\textbf{5.7} & \natline{యోగయుక్తో విశుద్ధాత్మా} \\
 & \natline{విజితాత్మా జితేన్ద్రియః |} \\
 & \natline{సర్వభూతాత్మభూతాత్మా} \\
 & \natline{కుర్వన్నపి న లిప్యతే ||}
\end{tabular}
\end{table}

\begin{table}[H]
\begin{tabular}{cl}
\textbf{5.8} & \natline{నైవ కిఞ్చిత్కరోమీతి} \\
 & \natline{యుక్తో మన్యేత తత్త్వవిత్ |} \\
 & \natline{పశ్యన్శృణ్వన్ స్పృశఞ్జిఘ్రన్} \\
 & \natline{అశ్నన్గచ్ఛన్స్వపన్శ్వసన్ ||}
\end{tabular}
\end{table}

\begin{table}[H]
\begin{tabular}{cl}
\textbf{5.9} & \natline{ప్రలపన్ విసృజన్ గృహ్ణన్} \\
 & \natline{ఉన్మిషన్నిమిషన్నపి |} \\
 & \natline{ఇన్ద్రియాణీన్ద్రియార్థేషు} \\
 & \natline{వర్తన్త ఇతి ధారయన్ ||}
\end{tabular}
\end{table}

\begin{table}[H]
\begin{tabular}{cl}
\textbf{5.10} & \natline{బ్రహ్మణ్యాధాయ కర్మాణి} \\
 & \natline{సఙ్గం త్యక్త్వా కరోతి యః |} \\
 & \natline{లిప్యతే న స పాపేన} \\
 & \natline{పద్మపత్రమివామ్భసా ||}
\end{tabular}
\end{table}

\begin{table}[H]
\begin{tabular}{cl}
\textbf{5.11} & \natline{కాయేన మనసా బుద్ధ్యా} \\
 & \natline{కేవలైరిన్ద్రియైరపి |} \\
 & \natline{యోగినః కర్మ కుర్వన్తి} \\
 & \natline{సఙ్గం త్యక్త్వాత్మశుద్ధయే ||}
\end{tabular}
\end{table}

\begin{table}[H]
\begin{tabular}{cl}
\textbf{5.12} & \natline{యుక్తః కర్మఫలం త్యక్త్వా} \\
 & \natline{శాన్తిమాప్నోతి నైష్ఠికీమ్ |} \\
 & \natline{అయుక్తః కామకారేణ} \\
 & \natline{ఫలే సక్తో నిబధ్యతే ||}
\end{tabular}
\end{table}

\begin{table}[H]
\begin{tabular}{cl}
\textbf{5.13} & \natline{సర్వకర్మాణి మనసా} \\
 & \natline{సన్న్యస్యాస్తే సుఖం వశీ |} \\
 & \natline{నవద్వారే పురే దేహీ} \\
 & \natline{నైవ కుర్వన్న కారయన్ ||}
\end{tabular}
\end{table}

\begin{table}[H]
\begin{tabular}{cl}
\textbf{5.14} & \natline{న కర్తృత్వం న కర్మాణి} \\
 & \natline{లోకస్య సృజతి ప్రభుః |} \\
 & \natline{న కర్మఫలసంయోగం} \\
 & \natline{స్వభావస్తు ప్రవర్తతే ||}
\end{tabular}
\end{table}

\begin{table}[H]
\begin{tabular}{cl}
\textbf{5.15} & \natline{నాదత్తే కస్యచిత్పాపం} \\
 & \natline{న చైవ సుకృతం విభుః |} \\
 & \natline{అజ్ఞానేనావృతం జ్ఞానం} \\
 & \natline{తేన ముహ్యన్తి జన్తవః ||}
\end{tabular}
\end{table}

\begin{table}[H]
\begin{tabular}{cl}
\textbf{5.16} & \natline{జ్ఞానేన తు తదజ్ఞానం} \\
 & \natline{యేషాం నాశితమాత్మనః |} \\
 & \natline{తేషామాదిత్యవజ్జ్ఞానం} \\
 & \natline{ప్రకాశయతి తత్పరమ్ ||}
\end{tabular}
\end{table}

\begin{table}[H]
\begin{tabular}{cl}
\textbf{5.17} & \natline{తద్బుద్ధయస్తదాత్మానః} \\
 & \natline{తన్నిష్ఠాస్తత్పరాయణాః |} \\
 & \natline{గచ్ఛన్త్యపునరావృత్తిం} \\
 & \natline{జ్ఞాననిర్ధూతకల్మషాః ||}
\end{tabular}
\end{table}

\begin{table}[H]
\begin{tabular}{cl}
\textbf{5.18} & \natline{విద్యావినయసమ్పన్నే} \\
 & \natline{బ్రాహ్మణే గవి హస్తిని |} \\
 & \natline{శుని చైవ శ్వపాకే చ} \\
 & \natline{పణ్డితాః సమదర్శినః ||}
\end{tabular}
\end{table}

\begin{table}[H]
\begin{tabular}{cl}
\textbf{5.19} & \natline{ఇహైవ తైర్జితః సర్గః} \\
 & \natline{యేషాం సామ్యే స్థితం మనః |} \\
 & \natline{నిర్దోషం హి సమం బ్రహ్మ} \\
 & \natline{తస్మాత్ బ్రహ్మణి తే స్థితాః ||}
\end{tabular}
\end{table}

\begin{table}[H]
\begin{tabular}{cl}
\textbf{5.20} & \natline{న ప్రహృష్యేత్ప్రియం ప్రాప్య} \\
 & \natline{నోద్విజేత్ప్రాప్య చాప్రియమ్ |} \\
 & \natline{స్థిరబుద్ధిరసమ్మూఢః} \\
 & \natline{బ్రహ్మవిత్ బ్రహ్మణి స్థితః ||}
\end{tabular}
\end{table}

\begin{table}[H]
\begin{tabular}{cl}
\textbf{5.21} & \natline{బాహ్యస్పర్శేష్వసక్తాత్మా} \\
 & \natline{విన్దత్యాత్మని యత్ సుఖమ్ |} \\
 & \natline{స బ్రహ్మయోగయుక్తాత్మా} \\
 & \natline{సుఖమక్షయమశ్నుతే ||}
\end{tabular}
\end{table}

\begin{table}[H]
\begin{tabular}{cl}
\textbf{5.22} & \natline{యే హి సంస్పర్శజా భోగాః} \\
 & \natline{దుఃఖయోనయ ఏవ తే |} \\
 & \natline{ఆద్యన్తవన్తః కౌన్తేయ} \\
 & \natline{న తేషు రమతే బుధః ||}
\end{tabular}
\end{table}

\begin{table}[H]
\begin{tabular}{cl}
\textbf{5.23} & \natline{శక్నోతీహైవ యః సోఢుం} \\
 & \natline{ప్రాక్ శరీరవిమోక్షణాత్ |} \\
 & \natline{కామక్రోధోద్భవం వేగం} \\
 & \natline{స యుక్తః స సుఖీ నరః ||}
\end{tabular}
\end{table}

\begin{table}[H]
\begin{tabular}{cl}
\textbf{5.24} & \natline{యోఽన్తఃసుఖోఽన్తరారామః} \\
 & \natline{తథాన్తర్జ్యోతిరేవ యః |} \\
 & \natline{స యోగీ బ్రహ్మనిర్వాణం} \\
 & \natline{బ్రహ్మభూతోఽధిగచ్ఛతి ||}
\end{tabular}
\end{table}

\begin{table}[H]
\begin{tabular}{cl}
\textbf{5.25} & \natline{లభన్తే బ్రహ్మనిర్వాణమ్} \\
 & \natline{ఋషయః క్షీణకల్మషాః |} \\
 & \natline{ఛిన్నద్వైధా యతాత్మానః} \\
 & \natline{సర్వభూతహితే రతాః ||}
\end{tabular}
\end{table}

\begin{table}[H]
\begin{tabular}{cl}
\textbf{5.26} & \natline{కామక్రోధవియుక్తానాం} \\
 & \natline{యతీనాం యతచేతసామ్ |} \\
 & \natline{అభితో బ్రహ్మనిర్వాణం} \\
 & \natline{వర్తతే విదితాత్మనామ్ ||}
\end{tabular}
\end{table}

\begin{table}[H]
\begin{tabular}{cl}
\textbf{5.27} & \natline{స్పర్శాన్ కృత్వా బహిర్బాహ్యాన్} \\
 & \natline{చక్షుశ్చైవాన్తరే భ్రువోః |} \\
 & \natline{ప్రాణాపానౌ సమౌ కృత్వా} \\
 & \natline{నాసాభ్యన్తరచారిణౌ ||}
\end{tabular}
\end{table}

\begin{table}[H]
\begin{tabular}{cl}
\textbf{5.28} & \natline{యతేన్ద్రియమనోబుద్ధిః} \\
 & \natline{మునిర్మోక్షపరాయణః |} \\
 & \natline{విగతేచ్ఛాభయక్రోధః} \\
 & \natline{యః సదా ముక్త ఏవ సః ||}
\end{tabular}
\end{table}

\begin{table}[H]
\begin{tabular}{cl}
\textbf{5.29} & \natline{భోక్తారం యజ్ఞతపసాం} \\
 & \natline{సర్వలోకమహేశ్వరమ్ |} \\
 & \natline{సుహృదం సర్వభూతానాం} \\
 & \natline{జ్ఞాత్వా మాం శాన్తిమృచ్ఛతి। ||}
\end{tabular}
\end{table}

