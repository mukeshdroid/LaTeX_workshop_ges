\begin{table}[H]
\begin{tabular}{cl}
\textbf{15.0} & \natline{ఓం శ్రీ పరమాత్మనే నమః} \\
 & \natline{అథ పఞ్చదశోఽధ్యాయః} \\
 & \natline{పురుషోత్తమప్రప్తియోగః}
\end{tabular}
\end{table}

\begin{table}[H]
\begin{tabular}{cl}
\textbf{15.1} & \natline{శ్రీభగవాన్ ఉవాచ} \\
 & \natline{ఊర్ధ్వమూలమధః* శాఖమ్} \\
 & \natline{అశ్వత్థం ప్రాహురవ్యయమ్ |} \\
 & \natline{ఛన్దాంసి యస్య పర్ణాని} \\
 & \natline{యస్తం వేద స వేదవిత్ ||}
\end{tabular}
\end{table}

\begin{table}[H]
\begin{tabular}{cl}
\textbf{15.2} & \natline{అధశ్చోర్ధ్వం ప్రసృతాస్తస్య శాఖాః} \\
 & \natline{గుణప్రవృద్ధా విషయప్రవాలాః |} \\
 & \natline{అధశ్చ మూలాన్యనుసన్తతాని} \\
 & \natline{కర్మానుబన్ధీని మనుష్యలోకే ||}
\end{tabular}
\end{table}

\begin{table}[H]
\begin{tabular}{cl}
\textbf{15.3} & \natline{న రూపమస్యేహ తథోపలభ్యతే} \\
 & \natline{నాన్తో న చాదిర్న చ సంప్రతిష్ఠా |} \\
 & \natline{అశ్వత్థమేనం సువిరూఢమూలమ్} \\
 & \natline{అసఙ్గశస్త్రేణ దృఢేన ఛిత్త్వా ||}
\end{tabular}
\end{table}

\begin{table}[H]
\begin{tabular}{cl}
\textbf{15.4} & \natline{తతః పదం తత్పరిమార్గితవ్యం} \\
 & \natline{యస్మిన్గతా న నివర్తన్తి భూయః |} \\
 & \natline{తమేవ చాద్యం పురుషం ప్రపద్యే} \\
 & \natline{యతః ప్రవృత్తిః ప్రసృతా పురాణీ ||}
\end{tabular}
\end{table}

\begin{table}[H]
\begin{tabular}{cl}
\textbf{15.5} & \natline{నిర్మానమోహా జితసఙ్గదోషాః} \\
 & \natline{అధ్యాత్మనిత్యా వినివృత్తకామాః |} \\
 & \natline{ద్వన్ద్వైర్విముక్తాః సుఖదుఃఖ సఞ్జ్ఞైః} \\
 & \natline{గచ్ఛన్త్యమూఢాః పదమవ్యయం తత్ ||}
\end{tabular}
\end{table}

\begin{table}[H]
\begin{tabular}{cl}
\textbf{15.6} & \natline{న తద్భాసయతే సూర్యః} \\
 & \natline{న శశాఙ్కో న పావకః |} \\
 & \natline{యద్గత్వా న నివర్తన్తే} \\
 & \natline{తద్ధామ పరమం మమ ||}
\end{tabular}
\end{table}

\begin{table}[H]
\begin{tabular}{cl}
\textbf{15.7} & \natline{మమైవాంశో జీవలోకే} \\
 & \natline{జీవభూతః సనాతనః |} \\
 & \natline{మనః షష్ఠానీన్ద్రియాణి} \\
 & \natline{ప్రకృతిస్థాని కర్షతి ||}
\end{tabular}
\end{table}

\begin{table}[H]
\begin{tabular}{cl}
\textbf{15.8} & \natline{శరీరం యదవాప్నోతి} \\
 & \natline{యచ్చాప్యుత్క్రామతీశ్వరః |} \\
 & \natline{గృహీత్వైతాని సంయాతి} \\
 & \natline{వాయుర్గన్ధానివాశయాత్ ||}
\end{tabular}
\end{table}

\begin{table}[H]
\begin{tabular}{cl}
\textbf{15.9} & \natline{శ్రోత్రం చక్షుః స్పర్శనం చ} \\
 & \natline{రసనం ఘ్రాణమేవ చ |} \\
 & \natline{అధిష్ఠాయ మనశ్చాయం} \\
 & \natline{విషయానుపసేవతే ||}
\end{tabular}
\end{table}

\begin{table}[H]
\begin{tabular}{cl}
\textbf{15.10} & \natline{ఉత్క్రామన్తం స్థితం వాఽపి} \\
 & \natline{భుఞ్జానం వా గుణాన్వితమ్ |} \\
 & \natline{విమూఢా నానుపశ్యన్తి} \\
 & \natline{పశ్యన్తి జ్ఞానచక్షుషః ||}
\end{tabular}
\end{table}

\begin{table}[H]
\begin{tabular}{cl}
\textbf{15.11} & \natline{యతన్తో యోగినశ్చైనం} \\
 & \natline{పశ్యన్త్యాత్మన్యవస్థితమ్ |} \\
 & \natline{యతన్తోఽప్యకృతాత్మానః} \\
 & \natline{నైనం పశ్యన్త్యచేతసః ||}
\end{tabular}
\end{table}

\begin{table}[H]
\begin{tabular}{cl}
\textbf{15.12} & \natline{యదాదిత్యగతం తేజః} \\
 & \natline{జగద్భాసయతేఽఖిలమ్ |} \\
 & \natline{యచ్చన్ద్రమసి యచ్చాగ్నౌ} \\
 & \natline{తత్తేజో విద్ధి మామకమ్ ||}
\end{tabular}
\end{table}

\begin{table}[H]
\begin{tabular}{cl}
\textbf{15.13} & \natline{గామావిశ్య చ భూతాని} \\
 & \natline{ధారయామ్యహమోజసా |} \\
 & \natline{పుష్ణామి చౌషధీః సర్వాః} \\
 & \natline{సోమో భూత్వా రసాత్మకః ||}
\end{tabular}
\end{table}

\begin{table}[H]
\begin{tabular}{cl}
\textbf{15.14} & \natline{అహం వైశ్వానరో భూత్వా} \\
 & \natline{ప్రాణినాం దేహమాశ్రితః |} \\
 & \natline{ప్రాణాపానసమాయుక్తః} \\
 & \natline{పచామ్యన్నం చతుర్విధమ్ ||}
\end{tabular}
\end{table}

\begin{table}[H]
\begin{tabular}{cl}
\textbf{15.15} & \natline{సర్వస్య చాహం హృది సన్నివిష్టః} \\
 & \natline{మత్తః స్మృతిర్జ్ఞానమపోహనం చ |} \\
 & \natline{వేదైశ్చ సర్వైరహమేవ వేద్యః} \\
 & \natline{వేదాన్తకృద్వేదవిదేవ చాహమ్ ||}
\end{tabular}
\end{table}

\begin{table}[H]
\begin{tabular}{cl}
\textbf{15.16} & \natline{ద్వావిమౌ పురుషౌ లోకే} \\
 & \natline{క్షరశ్చాక్షర ఏవ చ |} \\
 & \natline{క్షరః సర్వాణి భూతాని} \\
 & \natline{కూటస్థోఽక్షర ఉచ్యతే ||}
\end{tabular}
\end{table}

\begin{table}[H]
\begin{tabular}{cl}
\textbf{15.17} & \natline{ఉత్తమః పురుషస్త్వన్యః} \\
 & \natline{పరమాత్మేత్యుదాహృతః |} \\
 & \natline{యో లోకత్రయమావిశ్య} \\
 & \natline{బిభర్త్యవ్యయ ఈశ్వరః ||}
\end{tabular}
\end{table}

\begin{table}[H]
\begin{tabular}{cl}
\textbf{15.18} & \natline{యస్మాత్క్షరమతీతోఽహమ్} \\
 & \natline{అక్షరాదపి చోత్తమః |} \\
 & \natline{అతోఽస్మి లోకే వేదే చ} \\
 & \natline{ప్రథితః పురుషోత్తమః ||}
\end{tabular}
\end{table}

\begin{table}[H]
\begin{tabular}{cl}
\textbf{15.19} & \natline{యో మామేవమసమ్మూఢః} \\
 & \natline{జానాతి పురుషోత్తమమ్ |} \\
 & \natline{స సర్వవిద్భజతి మాం} \\
 & \natline{సర్వభావేన భారత ||}
\end{tabular}
\end{table}

\begin{table}[H]
\begin{tabular}{cl}
\textbf{15.20} & \natline{ఇతి గుహ్యతమం శాస్త్రమ్} \\
 & \natline{ఇదముక్తం మయాఽనఘ |} \\
 & \natline{ఏతద్బుద్ధ్వా బుద్ధిమాన్స్యాత్} \\
 & \natline{కృతకృత్యశ్చ భారత ||}
\end{tabular}
\end{table}

