\begin{table}[H]
\begin{tabular}{cl}
\textbf{12.0} & \romline{oṃ śrī paramātmane namaḥ} \\
 & \romline{atha dvādaśo'dhyāyaḥ} \\
 & \romline{bhakti-yogaḥ}
\end{tabular}
\end{table}

\begin{table}[H]
\begin{tabular}{cl}
\textbf{12.1} & \romline{arjuna uvāca} \\
 & \romline{evaṃ satatayuktā ye} \\
 & \romline{bhaktāstvāṃ paryupāsate |} \\
 & \romline{ye cāpyakṣaram-avyaktaṃ} \\
 & \romline{teṣāṃ ke yogavittamāḥ ||}
\end{tabular}
\end{table}

\begin{table}[H]
\begin{tabular}{cl}
\textbf{12.2} & \romline{śrī bhagavān uvāca} \\
 & \romline{mayyāveśya mano ye māṃ} \\
 & \romline{nityayuktā upāsate |} \\
 & \romline{śraddhayā parayopetāḥ} \\
 & \romline{te me yuktatamā matāḥ ||}
\end{tabular}
\end{table}

\begin{table}[H]
\begin{tabular}{cl}
\textbf{12.3} & \romline{ye tvakṣaram-anirdeśyam} \\
 & \romline{avyaktaṃ paryupāsate |} \\
 & \romline{sarvatragam-acintyam ca} \\
 & \romline{kūṭastha-macalam dhruvaṃ ||}
\end{tabular}
\end{table}

\begin{table}[H]
\begin{tabular}{cl}
\textbf{12.4} & \romline{sanniyamyendriya-grāmaṃ} \\
 & \romline{sarvatra samabuddhayaḥ |} \\
 & \romline{te prāpnuvanti māmeva} \\
 & \romline{sarvabhūtahite ratāḥ ||}
\end{tabular}
\end{table}

\begin{table}[H]
\begin{tabular}{cl}
\textbf{12.5} & \romline{kleśo'dhika-tarasteṣām} \\
 & \romline{avyaktāsakta-cetasām |} \\
 & \romline{avyaktā hi gatirduḥkhaṃ} \\
 & \romline{dehavad-bhiravāpyate ||}
\end{tabular}
\end{table}

\begin{table}[H]
\begin{tabular}{cl}
\textbf{12.6} & \romline{ye tu sarvāṇi karmāṇi} \\
 & \romline{mayi sannyasya matparāḥ |} \\
 & \romline{ananyenaiva yogena} \\
 & \romline{māṃ dhyāyanta upāsate ||}
\end{tabular}
\end{table}

\begin{table}[H]
\begin{tabular}{cl}
\textbf{12.7} & \romline{teṣāmahaṃ samuddhartā} \\
 & \romline{mṛtyu-saṃsāra-sāgarāt |} \\
 & \romline{bhavāmi nacirāt-pārtha} \\
 & \romline{mayyāveśita-cetasām ||}
\end{tabular}
\end{table}

\begin{table}[H]
\begin{tabular}{cl}
\textbf{12.8} & \romline{mayyeva mana ādhatsva} \\
 & \romline{mayi buddhiṃ niveśaya |} \\
 & \romline{nivasiṣyasi mayyeva} \\
 & \romline{ata ūrdhvaṃ na samśayaḥ ||}
\end{tabular}
\end{table}

\begin{table}[H]
\begin{tabular}{cl}
\textbf{12.9} & \romline{atha cittaṃ samādhātuṃ} \\
 & \romline{na śaknoṣi mayi sthiram |} \\
 & \romline{abhyāsayogena tataḥ} \\
 & \romline{māmicchāptuṃ dhanañjaya ||}
\end{tabular}
\end{table}

\begin{table}[H]
\begin{tabular}{cl}
\textbf{12.10} & \romline{abhyāse'pyasamartho'si} \\
 & \romline{mat-karma-paramo bhava |} \\
 & \romline{madarthamapi karmāṇi} \\
 & \romline{kurvan-siddhi-mavāpsyasi ||}
\end{tabular}
\end{table}

\begin{table}[H]
\begin{tabular}{cl}
\textbf{12.11} & \romline{athaitadapyaśakto'si} \\
 & \romline{kartuṃ mad-yogam-āśritaḥ |} \\
 & \romline{sarva-karma-phala-tyāgaṃ} \\
 & \romline{tataḥ kuru yatātmavān ||}
\end{tabular}
\end{table}

\begin{table}[H]
\begin{tabular}{cl}
\textbf{12.12} & \romline{śreyo hi·jñānam-abhyāsāt} \\
 & \romline{jñānāddhyānaṃ viśiṣyate |} \\
 & \romline{dhyānāt-karma-phala-tyāgaḥ} \\
 & \romline{tyāgācchāntiranantaram ||}
\end{tabular}
\end{table}

\begin{table}[H]
\begin{tabular}{cl}
\textbf{12.13} & \romline{adveṣṭā sarvabhūtānāṃ} \\
 & \romline{maitraḥ karuṇa eva ca |} \\
 & \romline{nirmamo nirahankāraḥ} \\
 & \romline{sama-duḥkha-sukhaḥ·kṣamī ||}
\end{tabular}
\end{table}

\begin{table}[H]
\begin{tabular}{cl}
\textbf{12.14} & \romline{santuṣṭaḥ satataṃ yogī} \\
 & \romline{yatātmā dṛḍha-niścayaḥ |} \\
 & \romline{mayyarpitamanobuddhiḥ} \\
 & \romline{yo madbhaktaḥ sa me priyaḥ ||}
\end{tabular}
\end{table}

\begin{table}[H]
\begin{tabular}{cl}
\textbf{12.15} & \romline{yasmānnodvijate lokaḥ} \\
 & \romline{lokānnodvijate ca yaḥ |} \\
 & \romline{harṣāmarṣa-bhayodvegaiḥ} \\
 & \romline{mukto yaḥ sa ca me priyaḥ ||}
\end{tabular}
\end{table}

\begin{table}[H]
\begin{tabular}{cl}
\textbf{12.16} & \romline{anapekṣaḥ śucirdakṣaḥ} \\
 & \romline{udāsīno gatavyathaḥ |} \\
 & \romline{sarvārambha-parityāgī} \\
 & \romline{yo madbhaktaḥ sa me priyaḥ ||}
\end{tabular}
\end{table}

\begin{table}[H]
\begin{tabular}{cl}
\textbf{12.17} & \romline{yo na hṛṣyati na·dveṣṭi} \\
 & \romline{na śocati na kāñkṣati |} \\
 & \romline{śubhāśubha-parityāgī} \\
 & \romline{bhaktimānyaḥ sa me priyaḥ ||}
\end{tabular}
\end{table}

\begin{table}[H]
\begin{tabular}{cl}
\textbf{12.18} & \romline{samaḥ śatrau ca mitre ca} \\
 & \romline{tathā mānāpamānayoḥ |} \\
 & \romline{śītoṣṇa-sukha-duḥkheṣu} \\
 & \romline{samaḥ sañga-vivarjitaḥ ||}
\end{tabular}
\end{table}

\begin{table}[H]
\begin{tabular}{cl}
\textbf{12.19} & \romline{tulya-nindā-stutirmaunī} \\
 & \romline{santuṣṭo yena kenacit |} \\
 & \romline{aniketaḥ sthiramatiḥ} \\
 & \romline{bhaktimānme priyo naraḥ ||}
\end{tabular}
\end{table}

\begin{table}[H]
\begin{tabular}{cl}
\textbf{12.20} & \romline{ye tu dharmyāmṛtamidaṃ} \\
 & \romline{yathoktaṃ paryupāsate |} \\
 & \romline{śraddadhānā matparamāḥ} \\
 & \romline{bhaktāste'tīva me priyāḥ ||}
\end{tabular}
\end{table}

