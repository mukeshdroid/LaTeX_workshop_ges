\begin{table}[H]
\begin{tabular}{cl}
\textbf{15.0} & \romline{oṃ śrī paramātmane namaḥ} \\
 & \romline{atha pañcadaśo'dhyāyaḥ} \\
 & \romline{puruṣottama-prapti-yogaḥ}
\end{tabular}
\end{table}

\begin{table}[H]
\begin{tabular}{cl}
\textbf{15.1} & \romline{śrī-bhagavān uvāca} \\
 & \romline{ūrdhva-mūlamadhaḥ* śākham} \\
 & \romline{aśvatthaṃ prāhuravyayam |} \\
 & \romline{chandāṃsi yasya parṇāni} \\
 & \romline{yastaṃ veda sa vedavit ||}
\end{tabular}
\end{table}

\begin{table}[H]
\begin{tabular}{cl}
\textbf{15.2} & \romline{adhaścordhvaṃ prasṛtāstasya śākhāḥ} \\
 & \romline{guṇa-pravṛddhā viṣaya-pravālāḥ |} \\
 & \romline{adhaśca mūlān-yanusantatāni} \\
 & \romline{karmānubandhīni manuṣyaloke ||}
\end{tabular}
\end{table}

\begin{table}[H]
\begin{tabular}{cl}
\textbf{15.3} & \romline{na rūpamasyeha tathopalabhyate} \\
 & \romline{nānto na cādirna ca saṃpratiṣṭhā |} \\
 & \romline{aśvatthamenaṃ suvirūḍhamūlam} \\
 & \romline{asaṅgaśastreṇa dṛḍhena chittvā ||}
\end{tabular}
\end{table}

\begin{table}[H]
\begin{tabular}{cl}
\textbf{15.4} & \romline{tataḥ padaṃ tatparimārgitavyaṃ} \\
 & \romline{yasmingatā na nivartanti bhūyaḥ |} \\
 & \romline{tameva cādyaṃ puruṣaṃ prapadye} \\
 & \romline{yataḥ pravṛttiḥ prasṛtā purāṇī ||}
\end{tabular}
\end{table}

\begin{table}[H]
\begin{tabular}{cl}
\textbf{15.5} & \romline{nirmānamohā jitasaṅgadoṣāḥ} \\
 & \romline{adhyātmanityā vinivṛttakāmāḥ |} \\
 & \romline{dvandvairvimuktāḥ sukhaduḥkha sañjñaiḥ} \\
 & \romline{gacchantyamūḍhāḥ pada-mavyayaṃ tat ||}
\end{tabular}
\end{table}

\begin{table}[H]
\begin{tabular}{cl}
\textbf{15.6} & \romline{na tadbhāsayate sūryaḥ} \\
 & \romline{na śaśāṅko na pāvakaḥ |} \\
 & \romline{yadgatvā na nivartante} \\
 & \romline{taddhāma paramaṃ mama ||}
\end{tabular}
\end{table}

\begin{table}[H]
\begin{tabular}{cl}
\textbf{15.7} & \romline{mamaivāṃśo jīvaloke} \\
 & \romline{jīvabhūtaḥ sanātanaḥ |} \\
 & \romline{manaḥ ṣaṣṭhānīndriyāṇi} \\
 & \romline{prakṛtisthāni karṣati ||}
\end{tabular}
\end{table}

\begin{table}[H]
\begin{tabular}{cl}
\textbf{15.8} & \romline{śarīraṃ yadavāpnoti} \\
 & \romline{yaccāpyut-krāmatīśvaraḥ |} \\
 & \romline{gṛhītvaitāni saṃyāti} \\
 & \romline{vāyurgandhā-nivāśayāt ||}
\end{tabular}
\end{table}

\begin{table}[H]
\begin{tabular}{cl}
\textbf{15.9} & \romline{śrotraṃ cakṣuḥ sparśanaṃ ca} \\
 & \romline{rasanaṃ ghrāṇameva ca |} \\
 & \romline{adhiṣṭhāya manaścāyaṃ} \\
 & \romline{viṣayānupasevate ||}
\end{tabular}
\end{table}

\begin{table}[H]
\begin{tabular}{cl}
\textbf{15.10} & \romline{utkrāmantaṃ sthitaṃ vā'pi} \\
 & \romline{bhuñjānaṃ vā guṇānvitam |} \\
 & \romline{vimūḍhā nānupaśyanti} \\
 & \romline{paśyanti·jñānacakṣuṣaḥ ||}
\end{tabular}
\end{table}

\begin{table}[H]
\begin{tabular}{cl}
\textbf{15.11} & \romline{yatanto yoginaścainaṃ} \\
 & \romline{paśyantyātmanyavasthitam |} \\
 & \romline{yatanto'pyakṛtātmānaḥ} \\
 & \romline{nainaṃ paśyantyacetasaḥ ||}
\end{tabular}
\end{table}

\begin{table}[H]
\begin{tabular}{cl}
\textbf{15.12} & \romline{yadādityagataṃ tejaḥ} \\
 & \romline{jagadbhāsayate'khilam |} \\
 & \romline{yaccandramasi yaccāgnau} \\
 & \romline{tattejo viddhi māmakam ||}
\end{tabular}
\end{table}

\begin{table}[H]
\begin{tabular}{cl}
\textbf{15.13} & \romline{gāmāviśya ca bhūtāni} \\
 & \romline{dhārayāmyahamojasā |} \\
 & \romline{puṣṇāmi cauṣadhīḥ sarvāḥ} \\
 & \romline{somo bhūtvā rasātmakaḥ ||}
\end{tabular}
\end{table}

\begin{table}[H]
\begin{tabular}{cl}
\textbf{15.14} & \romline{ahaṃ vaiśvānaro bhūtvā} \\
 & \romline{prāṇināṃ dehamāśritaḥ |} \\
 & \romline{prāṇāpānasamāyuktaḥ} \\
 & \romline{pacāmyannaṃ caturvidham ||}
\end{tabular}
\end{table}

\begin{table}[H]
\begin{tabular}{cl}
\textbf{15.15} & \romline{sarvasya cāhaṃ hṛdi sanniviṣṭaḥ} \\
 & \romline{mattaḥ smṛtirjñānamapohanaṃ ca |} \\
 & \romline{vedaiśca sarvairahameva vedyaḥ} \\
 & \romline{vedāntakṛdvedavideva cāham ||}
\end{tabular}
\end{table}

\begin{table}[H]
\begin{tabular}{cl}
\textbf{15.16} & \romline{dvāvimau puruṣau loke} \\
 & \romline{kṣaraścākṣara eva ca |} \\
 & \romline{kṣaraḥ sarvāṇi bhūtāni} \\
 & \romline{kūṭastho'kṣara ucyate ||}
\end{tabular}
\end{table}

\begin{table}[H]
\begin{tabular}{cl}
\textbf{15.17} & \romline{uttamaḥ puruṣastvanyaḥ} \\
 & \romline{paramātmetyudāhṛtaḥ |} \\
 & \romline{yo lokatrayamāviśya} \\
 & \romline{bibhartyavyaya īśvaraḥ ||}
\end{tabular}
\end{table}

\begin{table}[H]
\begin{tabular}{cl}
\textbf{15.18} & \romline{yasmātkṣaramatīto'ham} \\
 & \romline{akṣarādapi cottamaḥ |} \\
 & \romline{ato'smi loke vede ca} \\
 & \romline{prathitaḥ puruṣottamaḥ ||}
\end{tabular}
\end{table}

\begin{table}[H]
\begin{tabular}{cl}
\textbf{15.19} & \romline{yo māmevamasammūḍhaḥ} \\
 & \romline{jānāti puruṣottamam |} \\
 & \romline{sa sarvavidbhajati māṃ} \\
 & \romline{sarvabhāvena bhārata ||}
\end{tabular}
\end{table}

\begin{table}[H]
\begin{tabular}{cl}
\textbf{15.20} & \romline{iti guhyatamaṃ śāstram} \\
 & \romline{idamuktaṃ mayā'nagha |} \\
 & \romline{etadbuddhvā buddhimānsyāt} \\
 & \romline{kṛtakṛtyaśca bhārata ||}
\end{tabular}
\end{table}

