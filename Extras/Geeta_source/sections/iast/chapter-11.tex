\begin{table}[H]
\begin{tabular}{cl}
\textbf{11.0} & \romline{oṃ śrī paramātmane namaḥ} \\
 & \romline{atha ekādaśo'dhyāyaḥ} \\
 & \romline{viśvarūpa sandarśana yogaḥ}
\end{tabular}
\end{table}

\begin{table}[H]
\begin{tabular}{cl}
\textbf{11.1} & \romline{arjuna uvāca} \\
 & \romline{madanugrahāya paramaṃ} \\
 & \romline{guhya-madhyāt-masañjñitam |} \\
 & \romline{yattvayoktaṃ vacastena} \\
 & \romline{moho'yaṃ vigato mama ||}
\end{tabular}
\end{table}

\begin{table}[H]
\begin{tabular}{cl}
\textbf{11.2} & \romline{bhavāpyayau hi bhūtānāṃ} \\
 & \romline{śrutau vistaraśo mayā |} \\
 & \romline{tvattaḥ kamala-patrākṣa} \\
 & \romline{māhātmyamapi cāvyayam ||}
\end{tabular}
\end{table}

\begin{table}[H]
\begin{tabular}{cl}
\textbf{11.3} & \romline{evametad-yathā''ttha tvam} \\
 & \romline{ātmānaṃ parameśvara |} \\
 & \romline{draṣṭumicchāmi te rūpam} \\
 & \romline{aiśvaram puruṣottama ||}
\end{tabular}
\end{table}

\begin{table}[H]
\begin{tabular}{cl}
\textbf{11.4} & \romline{manyase yadi tacchakyaṃ} \\
 & \romline{mayā draṣṭumiti·prabho |} \\
 & \romline{yogeśvara tato me tvaṃ} \\
 & \romline{darśayātmāna-mavyayam ||}
\end{tabular}
\end{table}

\begin{table}[H]
\begin{tabular}{cl}
\textbf{11.5} & \romline{śrī bhagavānuvāca} \\
 & \romline{paśya me pārtha rūpāṇi} \\
 & \romline{śataśo'tha sahasraśaḥ |} \\
 & \romline{nānāvidhāni divyāni} \\
 & \romline{nānāvarṇākṛtīni ca ||}
\end{tabular}
\end{table}

\begin{table}[H]
\begin{tabular}{cl}
\textbf{11.6} & \romline{paśyā-dityān-vasūnrudrān} \\
 & \romline{aśvinau marutastathā |} \\
 & \romline{bahūnyadṛṣṭa-pūrvāṇi} \\
 & \romline{paśyāścaryāṇi bhārata ||}
\end{tabular}
\end{table}

\begin{table}[H]
\begin{tabular}{cl}
\textbf{11.7} & \romline{ihaikasthaṃ jagatkṛtsnaṃ} \\
 & \romline{paśyādya sacarācaram |} \\
 & \romline{mama dehe guḍākeśa} \\
 & \romline{yaccānyat draṣṭumicchasi ||}
\end{tabular}
\end{table}

\begin{table}[H]
\begin{tabular}{cl}
\textbf{11.8} & \romline{na tu māṃ śakyase draṣṭum} \\
 & \romline{anenaiva·svacakṣuṣā |} \\
 & \romline{divyaṃ dadāmi te cakṣuḥ} \\
 & \romline{paśya me yogamaiśvaram ||}
\end{tabular}
\end{table}

\begin{table}[H]
\begin{tabular}{cl}
\textbf{11.9} & \romline{sañjaya uvāca} \\
 & \romline{evamuktvā tato rājan} \\
 & \romline{mahā-yogeśvaro hariḥ |} \\
 & \romline{darśayā-māsa pārthāya} \\
 & \romline{paramaṃ rūpa-maiśvaram ||}
\end{tabular}
\end{table}

\begin{table}[H]
\begin{tabular}{cl}
\textbf{11.10} & \romline{aneka-vaktra-nayanam} \\
 & \romline{anekād-bhuta-darśanam |} \\
 & \romline{aneka-divyā-bharaṇaṃ} \\
 & \romline{divyā-nekodya-tāyudham ||}
\end{tabular}
\end{table}

\begin{table}[H]
\begin{tabular}{cl}
\textbf{11.11} & \romline{divya-mālyāmbara-dharaṃ} \\
 & \romline{divya-gandhānulepanam |} \\
 & \romline{sarvāścarya-mayaṃ devam} \\
 & \romline{anantaṃ viśvatomukham ||}
\end{tabular}
\end{table}

\begin{table}[H]
\begin{tabular}{cl}
\textbf{11.12} & \romline{divi sūrya-sahasrasya} \\
 & \romline{bhaved-yuga-padutthitā |} \\
 & \romline{yadi bhāḥ sadṛśī sā syāt} \\
 & \romline{bhāsastasya mahātmanaḥ ||}
\end{tabular}
\end{table}

\begin{table}[H]
\begin{tabular}{cl}
\textbf{11.13} & \romline{tatrai-kasthaṃ jagat-kṛtsnaṃ} \\
 & \romline{pravibhakta-manekadhā |} \\
 & \romline{apaśyad-devadevasya} \\
 & \romline{śarīre pāṇḍavastadā ||}
\end{tabular}
\end{table}

\begin{table}[H]
\begin{tabular}{cl}
\textbf{11.14} & \romline{tataḥ sa vismayāviṣṭaḥ} \\
 & \romline{hṛṣṭaromā dhanañjayaḥ |} \\
 & \romline{praṇamya śirasā devaṃ} \\
 & \romline{kṛtāñ-jalira-bhāṣata ||}
\end{tabular}
\end{table}

\begin{table}[H]
\begin{tabular}{cl}
\textbf{11.15} & \romline{arjuna uvāca} \\
 & \romline{paśyāmi devāṃstava deva dehe} \\
 & \romline{sarvāṃstathā bhūta-viśeṣa-saṅghān |} \\
 & \romline{brahmāṇa-mīśaṃ kamalā-sanastham} \\
 & \romline{ṛṣīṃśca sarvā-nuragāṃśca divyān ||}
\end{tabular}
\end{table}

\begin{table}[H]
\begin{tabular}{cl}
\textbf{11.16} & \romline{aneka-bāhū-dara-vaktra-netraṃ} \\
 & \romline{paśyāmi·tvā sarvato'nantarūpam |} \\
 & \romline{nāntaṃ na madhyaṃ na punastavādiṃ} \\
 & \romline{paśyāmi viśveśvara viśvarūpa ||}
\end{tabular}
\end{table}

\begin{table}[H]
\begin{tabular}{cl}
\textbf{11.17} & \romline{kirīṭinaṃ gadinaṃ cakriṇaṃ ca} \\
 & \romline{tejorāśiṃ sarvato dīptimantam |} \\
 & \romline{paśyāmi·tvāṃ durni-rīkṣyaṃ samantāt} \\
 & \romline{dīptā-nalārka-dyuti-maprameyam ||}
\end{tabular}
\end{table}

\begin{table}[H]
\begin{tabular}{cl}
\textbf{11.18} & \romline{tvamakṣaraṃ paramaṃ veditavyaṃ} \\
 & \romline{tvamasya viśvasya paraṃ nidhānam |} \\
 & \romline{tva-mavyayaḥ śāśvata-dharma-goptā} \\
 & \romline{sanātanastvaṃ puruṣo mato me ||}
\end{tabular}
\end{table}

\begin{table}[H]
\begin{tabular}{cl}
\textbf{11.19} & \romline{anādi-madhyānta-mananta-vīryam} \\
 & \romline{ananta-bāhuṃ śaśi-sūrya-netram |} \\
 & \romline{paśyāmi·tvāṃ dīptahutā-śavaktraṃ} \\
 & \romline{svatejasā viśvamidaṃ tapantam ||}
\end{tabular}
\end{table}

\begin{table}[H]
\begin{tabular}{cl}
\textbf{11.20} & \romline{dyāvā-pṛthiv-yorida-mantaraṃ hi} \\
 & \romline{vyāptaṃ tva-yaikena diśaśca sarvāḥ |} \\
 & \romline{dṛṣṭvād-bhutaṃ rūpamidaṃ tavograṃ} \\
 & \romline{lokatrayaṃ pravyathitaṃ mahātman ||}
\end{tabular}
\end{table}

\begin{table}[H]
\begin{tabular}{cl}
\textbf{11.21} & \romline{amī hi·tvā surasaṅghā viśanti} \\
 & \romline{kecid-bhītāḥ prāñjalayo gṛṇanti |} \\
 & \romline{svastī-tyuktvā maharṣisiddha-saṅghāḥ} \\
 & \romline{stuvanti·tvāṃ stutibhiḥ puṣkalābhiḥ ||}
\end{tabular}
\end{table}

\begin{table}[H]
\begin{tabular}{cl}
\textbf{11.22} & \romline{rudrādityā vasavo ye ca sādhyāḥ} \\
 & \romline{viśve'śvinau marutaścoṣmapāśca |} \\
 & \romline{gandharva-yakṣā-surasiddha-saṅghāḥ} \\
 & \romline{vīkṣante tvāṃ vismitāścaiva sarve ||}
\end{tabular}
\end{table}

\begin{table}[H]
\begin{tabular}{cl}
\textbf{11.23} & \romline{rūpaṃ mahatte bahuvaktra netraṃ} \\
 & \romline{mahābāho bahubāhū-rupādam |} \\
 & \romline{bahūdaraṃ bahudaṃṣṭrā-karālaṃ} \\
 & \romline{dṛṣṭvā lokāḥ pravyathitā-stathā'ham ||}
\end{tabular}
\end{table}

\begin{table}[H]
\begin{tabular}{cl}
\textbf{11.24} & \romline{nabhaḥ spṛśaṃ dīpta-manekavarṇaṃ} \\
 & \romline{vyāttā-nanaṃ dīpta-viśāla-netram |} \\
 & \romline{dṛṣṭvā hi·tvāṃ pravyathitān-tarātmā} \\
 & \romline{dhṛtiṃ na vindāmi śamaṃ ca viṣṇo ||}
\end{tabular}
\end{table}

\begin{table}[H]
\begin{tabular}{cl}
\textbf{11.25} & \romline{daṃṣṭrā-karālāni ca te mukhāni} \\
 & \romline{dṛṣṭvaiva kālānala-sannibhāni |} \\
 & \romline{diśo na jāne na labhe ca śarma} \\
 & \romline{prasīda deveśa jagannivāsa ||}
\end{tabular}
\end{table}

\begin{table}[H]
\begin{tabular}{cl}
\textbf{11.26} & \romline{amī ca tvāṃ dhṛtarāṣ-ṭrasya putrāḥ} \\
 & \romline{sarve sahai-vāvani-pāla-saṅghaiḥ |} \\
 & \romline{bhīṣmo droṇaḥ sūta-putra-stathā'sau} \\
 & \romline{sahāsma-dīyairapi yodha-mukhyaiḥ ||}
\end{tabular}
\end{table}

\begin{table}[H]
\begin{tabular}{cl}
\textbf{11.27} & \romline{vaktrāṇi te tvara-māṇā viśanti} \\
 & \romline{daṃṣṭrā-karālāni bhayā-nakāni |} \\
 & \romline{kecid-vilagnā daśanān-tareṣu} \\
 & \romline{sandṛśyante cūrṇitai-rutta-māṅgaiḥ ||}
\end{tabular}
\end{table}

\begin{table}[H]
\begin{tabular}{cl}
\textbf{11.28} & \romline{yathā nadīnāṃ bahavo'mbuvegāḥ} \\
 & \romline{samudra-mevābhi-mukhā dravanti |} \\
 & \romline{tathā tavāmī naralo-kavīrāḥ} \\
 & \romline{viśanti vaktrāṇya-bhivij-valanti ||}
\end{tabular}
\end{table}

\begin{table}[H]
\begin{tabular}{cl}
\textbf{11.29} & \romline{yathā pradīptaṃ jvalanaṃ pataṅgāḥ} \\
 & \romline{viśanti nāśāya samṛddha-vegāḥ |} \\
 & \romline{tathaiva nāśāya viśanti lokāḥ} \\
 & \romline{tavāpi vaktrāṇi samṛddha-vegāḥ ||}
\end{tabular}
\end{table}

\begin{table}[H]
\begin{tabular}{cl}
\textbf{11.30} & \romline{lelihyase grasamānaḥ samantāt} \\
 & \romline{lokān-samagrān-vadanair-jvaladbhiḥ |} \\
 & \romline{tejobhirāpūrya jagat-samagraṃ} \\
 & \romline{bhāsas-tavogrāḥ pratapanti viṣṇo ||}
\end{tabular}
\end{table}

\begin{table}[H]
\begin{tabular}{cl}
\textbf{11.31} & \romline{ākhyāhi me ko bhavā-nugra-rūpaḥ} \\
 & \romline{namo'stu te deva-vara·prasīda |} \\
 & \romline{vijñātu-micchāmi bhavanta-mādyaṃ} \\
 & \romline{na hi·prajānāmi tava·pravṛttim ||}
\end{tabular}
\end{table}

\begin{table}[H]
\begin{tabular}{cl}
\textbf{11.32} & \romline{śrī bhagavānuvāca} \\
 & \romline{kālo'smi lokakṣayakṛt-pravṛddhaḥ} \\
 & \romline{lokān-samāhartu-miha·pravṛttaḥ |} \\
 & \romline{ṛte'pi·tvā na bhaviṣyanti sarve} \\
 & \romline{ye'vasthitāḥ pratya-nīkeṣu yodhāḥ ||}
\end{tabular}
\end{table}

\begin{table}[H]
\begin{tabular}{cl}
\textbf{11.33} & \romline{tasmāttva-muttiṣṭha yaśo labhasva} \\
 & \romline{jitvā śatrūn-bhuṅkṣva rājyaṃ samṛddham |} \\
 & \romline{mayaivaite nihatāḥ pūrvameva} \\
 & \romline{nimitta-mātraṃ bhava savyasācin ||}
\end{tabular}
\end{table}

\begin{table}[H]
\begin{tabular}{cl}
\textbf{11.34} & \romline{droṇaṃ ca bhīṣmaṃ ca jayadrathaṃ ca} \\
 & \romline{karṇaṃ tathānyānapi yodhavīrān |} \\
 & \romline{mayā hatāṃstvaṃ jahi mā vyathiṣṭhāḥ} \\
 & \romline{yudhyasva jetāsi raṇe sapatnān ||}
\end{tabular}
\end{table}

\begin{table}[H]
\begin{tabular}{cl}
\textbf{11.35} & \romline{sañjaya uvāca} \\
 & \romline{etacchrutvā vacanaṃ keśavasya} \\
 & \romline{kṛtāñjalir-vepamānaḥ kirīṭī |} \\
 & \romline{namaskṛtvā bhūya evāha kṛṣṇaṃ} \\
 & \romline{sagad-gadaṃ bhīta-bhītaḥ praṇamya ||}
\end{tabular}
\end{table}

\begin{table}[H]
\begin{tabular}{cl}
\textbf{11.36} & \romline{arjuna uvāca} \\
 & \romline{sthāne hṛṣīkeśa tava·prakīrtyā} \\
 & \romline{jagat-prahṛṣya-tyanurajyate ca |} \\
 & \romline{rakṣāṃsi bhītāni diśo dravanti} \\
 & \romline{sarve namasyanti ca siddha-saṅghāḥ ||}
\end{tabular}
\end{table}

\begin{table}[H]
\begin{tabular}{cl}
\textbf{11.37} & \romline{kasmācca te na nameran-mahātman} \\
 & \romline{garīyase brahmaṇo'pyādi-kartre |} \\
 & \romline{ananta deveśa jagannivāsa} \\
 & \romline{tvamakṣaraṃ sadasat-tatparaṃ yat ||}
\end{tabular}
\end{table}

\begin{table}[H]
\begin{tabular}{cl}
\textbf{11.38} & \romline{tvamādi-devaḥ puruṣaḥ purāṇaḥ} \\
 & \romline{tvamasya viśvasya paraṃ nidhānam |} \\
 & \romline{vettā'si vedyaṃ ca paraṃ ca dhāma} \\
 & \romline{tvayā tataṃ viśva-manantarūpa ||}
\end{tabular}
\end{table}

\begin{table}[H]
\begin{tabular}{cl}
\textbf{11.39} & \romline{vāyur-yamo'gnir-varuṇaḥ śaśāṅkaḥ} \\
 & \romline{prajā-patistvaṃ prapitā-mahaśca |} \\
 & \romline{namo namaste'stu sahasra-kṛtvaḥ} \\
 & \romline{punaśca bhūyo'pi namo namaste ||}
\end{tabular}
\end{table}

\begin{table}[H]
\begin{tabular}{cl}
\textbf{11.40} & \romline{namaḥ purastā-datha·pṛṣṭha-taste} \\
 & \romline{namo'stu te sarvata eva sarva |} \\
 & \romline{ananta-vīryā-mita-vikrama-stvaṃ} \\
 & \romline{sarvaṃ samāpnoṣi tato'si sarvaḥ ||}
\end{tabular}
\end{table}

\begin{table}[H]
\begin{tabular}{cl}
\textbf{11.41} & \romline{sakheti matvā prasabhaṃ yaduktaṃ} \\
 & \romline{he kṛṣṇa he yādava he sakheti |} \\
 & \romline{ajānatā mahimānaṃ tavedaṃ} \\
 & \romline{mayā pramādāt-praṇayena vā'pi ||}
\end{tabular}
\end{table}

\begin{table}[H]
\begin{tabular}{cl}
\textbf{11.42} & \romline{yaccā-pahāsārthama-satkṛto'si} \\
 & \romline{vihāra-śayyā-sanabho-janeṣu |} \\
 & \romline{eko'thavā-pyacyuta tatsamakṣaṃ} \\
 & \romline{tatkṣāmaye tvā-mahama-prameyam ||}
\end{tabular}
\end{table}

\begin{table}[H]
\begin{tabular}{cl}
\textbf{11.43} & \romline{pitāsi lokasya carācarasya} \\
 & \romline{tvamasya pūjyaśca gurur-garīyān |} \\
 & \romline{na tvatsamo'styabhya-dhikaḥ kuto'nyaḥ} \\
 & \romline{lokatraye'pya-pratima-prabhāva ||}
\end{tabular}
\end{table}

\begin{table}[H]
\begin{tabular}{cl}
\textbf{11.44} & \romline{tasmāt-praṇamya·praṇidhāya kāyaṃ} \\
 & \romline{prasādaye tvā-maha-mīśa-mīḍyam |} \\
 & \romline{piteva putrasya sakheva sakhyuḥ} \\
 & \romline{priyaḥ priyā-yārhasi deva soḍhum ||}
\end{tabular}
\end{table}

\begin{table}[H]
\begin{tabular}{cl}
\textbf{11.45} & \romline{adṛṣṭa-pūrvaṃ hṛṣito'smi dṛṣṭvā} \\
 & \romline{bhayena ca·pravyathitaṃ mano me |} \\
 & \romline{tadeva me darśaya devarūpaṃ} \\
 & \romline{prasīda deveśa jagannivāsa ||}
\end{tabular}
\end{table}

\begin{table}[H]
\begin{tabular}{cl}
\textbf{11.46} & \romline{kirīṭinaṃ gadinaṃ cakrahastam} \\
 & \romline{icchāmi·tvāṃ draṣṭumahaṃ tathaiva |} \\
 & \romline{tenaiva rūpeṇa caturbhujena} \\
 & \romline{sahasrabāho bhava viśvamūrte ||}
\end{tabular}
\end{table}

\begin{table}[H]
\begin{tabular}{cl}
\textbf{11.47} & \romline{śrī bhagavānuvāca} \\
 & \romline{mayā prasannena tavārjunedaṃ} \\
 & \romline{rūpaṃ paraṃ darśita-mātmayogāt |} \\
 & \romline{tejomayaṃ viśva-mananta-mādyaṃ} \\
 & \romline{yanme tvadanyena na dṛṣṭapūrvam ||}
\end{tabular}
\end{table}

\begin{table}[H]
\begin{tabular}{cl}
\textbf{11.48} & \romline{na veda-yajñā-dhyaya-nairna dānaiḥ} \\
 & \romline{na ca kriyābhirna tapo-bhirugraiḥ |} \\
 & \romline{evaṃrūpaḥ śakya ahaṃ nṛloke} \\
 & \romline{draṣṭuṃ tvadanyena kurupravīra ||}
\end{tabular}
\end{table}

\begin{table}[H]
\begin{tabular}{cl}
\textbf{11.49} & \romline{mā te vyathā mā ca vimūḍha-bhāvaḥ} \\
 & \romline{dṛṣṭvā rūpaṃ ghoramīdṛṅ-mamedam |} \\
 & \romline{vyapeta-bhīḥ prīta-manāḥ punastvaṃ} \\
 & \romline{tadeva me rūpamidaṃ prapaśya ||}
\end{tabular}
\end{table}

\begin{table}[H]
\begin{tabular}{cl}
\textbf{11.50} & \romline{sañjaya uvāca} \\
 & \romline{ityarjunaṃ vāsudevas-tathoktvā} \\
 & \romline{svakaṃ rūpaṃ darśayāmāsa bhūyaḥ |} \\
 & \romline{āśvāsayā-māsa ca bhītamenaṃ} \\
 & \romline{bhūtvā punaḥ saumya-vapur-mahātmā ||}
\end{tabular}
\end{table}

\begin{table}[H]
\begin{tabular}{cl}
\textbf{11.51} & \romline{arjuna uvāca} \\
 & \romline{dṛṣṭvedaṃ mānuṣaṃ rūpaṃ} \\
 & \romline{tava saumyaṃ janārdana |} \\
 & \romline{idānīmasmi saṃvṛttaḥ} \\
 & \romline{sacetāḥ prakṛtiṃ gataḥ ||}
\end{tabular}
\end{table}

\begin{table}[H]
\begin{tabular}{cl}
\textbf{11.52} & \romline{śrī bhagavānuvāca} \\
 & \romline{sudur-darśa-midaṃ rūpaṃ} \\
 & \romline{dṛṣṭa-vānasi yanmama |} \\
 & \romline{devā apyasya rūpasya} \\
 & \romline{nityaṃ darśa-nakāṅkṣiṇaḥ ||}
\end{tabular}
\end{table}

\begin{table}[H]
\begin{tabular}{cl}
\textbf{11.53} & \romline{nāhaṃ vedairna tapasā} \\
 & \romline{na dānena na cejyayā |} \\
 & \romline{śakya evaṃ-vidho draṣṭuṃ} \\
 & \romline{dṛṣṭa-vānasi māṃ yathā ||}
\end{tabular}
\end{table}

\begin{table}[H]
\begin{tabular}{cl}
\textbf{11.54} & \romline{bhaktyā tva-nanyayā śakyaḥ} \\
 & \romline{ahamevaṃ-vidho'rjuna |} \\
 & \romline{jñātuṃ draṣṭuṃ ca tattvena} \\
 & \romline{praveṣṭuṃ ca parantapa ||}
\end{tabular}
\end{table}

\begin{table}[H]
\begin{tabular}{cl}
\textbf{11.55} & \romline{matkarma-kṛn-matparamaḥ} \\
 & \romline{madbhaktaḥ saṅga-varjitaḥ |} \\
 & \romline{nirvairaḥ sarva-bhūteṣu} \\
 & \romline{yaḥ sa māmeti pāṇḍava ||}
\end{tabular}
\end{table}

